\section{Extended Kalman filter}
\label{sec:ekf}
\index{nonlinear control!Extended Kalman filter}
\index{state-space observers!Extended Kalman filter}

In this book, we have covered the Kalman filter, which is the optimal unbiased
estimator for linear \glspl{system}. One method for extending it to nonlinear
systems is the extended Kalman filter.

The extended Kalman filter (EKF) \glslink{linearization}{linearizes} the
dynamics and measurement models during the prediction and correction steps
respectively. Then, the linear Kalman filter equations are used to compute the
error covariance matrix $\mat{P}$ and Kalman gain matrix $\mat{K}$.

Theorem \ref{thm:extended_kalman_filter} shows the predict and update steps for
an extended Kalman filter at the $k^{th}$ timestep.
\begin{theorem}[Extended Kalman filter]
  \label{thm:extended_kalman_filter}
  \begin{align}
    \text{Predict step} \nonumber \\
    \mat{A} &= \left.\frac{\partial f(\mat{x}, \mat{u})}{\partial \mat{x}}
               \right|_{\hat{\mat{x}}_k^+, \mat{u}_k}
      \quad\text{(linearize $f(\mat{x}, \mat{u})$)} \\
    \mat{A}_k &= e^{\mat{A}T} \qquad\text{(discretize $\mat{A}$)} \\
    \hat{\mat{x}}_{k+1}^- &= f(\hat{\mat{x}}_k^+, \mat{u}_k) \\
    \mat{P}_{k+1}^- &= \mat{A}_k \mat{P}_k^- \mat{A}_k\T + \mat{Q}_k \\
    \text{Update step} \nonumber \\
    \mat{C}_{k+1} &= \left.\frac{\partial h(\mat{x}, \mat{u})}{\partial \mat{x}}
                     \right|_{\hat{\mat{x}}_{k+1}^-, \mat{u}_{k+1}}
      \quad\text{(linearize $h(\mat{x}, \mat{u})$)} \\
    \mat{K}_{k+1} &= \mat{P}_{k+1}^- \mat{C}_{k+1}\T
      (\mat{C}_{k+1}\mat{P}_{k+1}^- \mat{C}_{k+1}\T + \mat{R}_{k+1})^{-1} \\
    \hat{\mat{x}}_{k+1}^+ &=
      \hat{\mat{x}}_{k+1}^- + \mat{K}_{k+1}(\mat{y}_{k+1} -
      h(\hat{\mat{x}}_{k+1}^-, \mat{u}_{k+1})) \\
    \mat{P}_{k+1}^+ &= (\mat{I} - \mat{K}_{k+1}\mat{C}_{k+1})\mat{P}_{k+1}^-
  \end{align}
  \begin{figurekey}
    \begin{tabular}{llll}
      $f(\mat{x}, \mat{u})$ & continuous dynamics model & $\hat{\mat{x}}$ & state estimate vector \\
      $h(\mat{x}, \mat{u})$ & measurement model & $\mat{u}$ & input vector \\
      $T$ & sample timestep duration & $\mat{y}$ & output vector \\
      $\mat{P}$ & error covariance matrix & $\mat{Q}$ & process noise covariance \\
      $\mat{K}$ & Kalman gain matrix & $\mat{R}$ & measurement noise covariance \\
    \end{tabular}
  \end{figurekey}

  where a superscript of minus denotes \textit{a priori} and plus denotes
  \textit{a posteriori} estimate (before and after update respectively).
\end{theorem}
\begin{remark}
  To implement a discrete time Kalman filter from a continuous model, the model
  and continuous time $\mat{Q}$ and $\mat{R}$ matrices can be
  \glslink{discretization}{discretized} using theorem
  \ref{thm:linear_system_zoh}.
\end{remark}
\begin{booktable}
  \begin{tabular}{|ll|ll|}
    \hline
    \rowcolor{headingbg}
    \textbf{Matrix} & \textbf{Rows $\times$ Columns} &
    \textbf{Matrix} & \textbf{Rows $\times$ Columns} \\
    \hline
    $\mat{A}$ & states $\times$ states & $\hat{\mat{x}}$ & states $\times$ 1 \\
    $\mat{C}$ & outputs $\times$ states & $\mat{u}$ & inputs $\times$ 1 \\
    $\mat{P}$ & states $\times$ states & $\mat{y}$ & outputs $\times$ 1 \\
    $\mat{K}$ & states $\times$ outputs & $\mat{Q}$ & states $\times$ states \\
              &                         & $\mat{R}$ & outputs $\times$ outputs \\
    \hline
  \end{tabular}
  \caption{Extended Kalman filter matrix dimensions}
\end{booktable}
