\chapterimage{banana-slug.jpg}{A sample of the Santa Cruz banana distribution
  \cite{bib:banana_distribution}}

\chapter{Stochastic control theory}
\label{ch:stochastic_control_theory}

Stochastic control theory is a subfield of control theory that deals with the
existence of uncertainty either in observations or in noise that drives the
evolution of a \gls{system}. We assign probability distributions to this random
noise and aim to achieve a desired control task despite the presence of this
uncertainty.

Stochastic optimal control is concerned with doing this with minimum cost
defined by some cost functional, like we did with LQR earlier. First, we'll
cover the basics of probability and how we represent linear stochastic
\glspl{system} in state-space representation. Then, we'll derive an optimal
estimator using this knowledge, the Kalman filter, and demonstrate creative
applications of the Kalman filter theory.

This will be a math-heavy introduction, so we recommend reading \textit{Kalman
and Bayesian Filters in Python} by Roger Labbe first
\cite{bib:kalman_and_bayesian_filters_in_python}.

\renewcommand*{\chapterpath}{\partpath/stochastic-control-theory}
\input{\chapterpath/terminology}
\input{\chapterpath/state-observers}
\section{Introduction to probability}
\index{probability}

Now we'll begin establishing probability concepts we need to describe and
manipulate stochastic systems.

\subsection{Random variables}
\index{probability!random variables}

A random variable is a variable whose values are the outcomes of a random
phenomenon, like dice rolls or noisy process measurements. As such, a random
variable is defined as a function that maps the outcomes of an unpredictable
process to numerical quantities. A particular output of this function is called
a sample. The sample space is the set of possible values taken by the random
variable.

\index{probability!probability density function}
A probability density function (PDF) is a function of the random variable whose
value at a given sample (measured value) in the sample space (the range of
possible measured values) is the probability of that sample occurrring. The area
under the function over a range gives the probability that the sample falls
within that range. Let $x$ be a random variable, and let $p(x)$ denote the
probability density function of $x$. The probability that the value of $x$ will
be in the interval $x \in [x_1, x_1 + dx]$ is $p(x_1) \,dx$. In other words, the
probability is the area under the PDF within the region $[x_1, x_1 + dx]$ (see
figure \ref{fig:pdf}).
\begin{svg}{build/\chapterpath/pdf}
  \caption{Probability density function}
  \label{fig:pdf}
\end{svg}

A probability of zero means that the sample will not occur and a probability of
one means that the sample will always occur. Probability density functions
require that no probabilities are negative and that the sum of all probabilities
is $1$. If the probabilities sum to $1$, that means one of those outcomes
\textit{must} happen. In other words,
\begin{equation*}
  p(x) \geq 0, \int_{-\infty}^\infty p(x) \,dx = 1
\end{equation*}

or given that the probability of a given sample is greater than or equal to
zero, the sum of probabilities for all possible input values is equal to one.

\subsection{Expected value}
\index{probability!expected value}

Expected value or expectation is a weighted average of the values the PDF can
produce where the weight for each value is the probability of that value
occurring. This can be written mathematically as
\begin{equation*}
  \overline{x} = E[x] = \int_{-\infty}^\infty x \,p(x) \,dx
\end{equation*}

The expectation can be applied to random functions as well as random variables.
\begin{equation*}
  E[f(x)] = \int_{-\infty}^\infty f(x) \,p(x) \,dx
\end{equation*}

The mean of a random variable is denoted by an overbar (e.g., $\overline{x}$)
pronounced x-bar. The expectation of the difference between a random variable
and its mean $x - \overline{x}$ converges to zero. In other words, the
expectation of a random variable is its mean. Here's a proof.
\begin{align*}
  E[x - \overline{x}] &= \int_{-\infty}^\infty (x - \overline{x}) \,p(x) \,dx \\
  E[x - \overline{x}] &= \int_{-\infty}^\infty x \, p(x) \,dx -
    \int_{-\infty}^\infty \overline{x} \,p(x) \,dx \\
  E[x - \overline{x}] &= \int_{-\infty}^\infty x \,p(x) \,dx -
    \overline{x} \int_{-\infty}^\infty p(x) \,dx \\
  E[x - \overline{x}] &= \overline{x} - \overline{x} \cdot 1 \\
  E[x - \overline{x}] &= 0 \\
\end{align*}

\subsection{Variance}
\index{probability!variance}

Informally, variance is a measure of how far the outcome of a random variable
deviates from its mean. Later, we will use variance to quantify how confident we
are in the estimate of a random variable; we can't know the true value of that
variable without randomness, but we can give a bound on its randomness.
\begin{equation*}
  \var(x) = \sigma^2 = E[(x - \overline{x})^2] =
    \int_{-\infty}^{\infty} (x - \overline{x})^2 \,p(x) \,dx
\end{equation*}

The standard deviation is the square root of the variance.
\begin{equation*}
  \std[x] = \sigma = \sqrt{\var(x)}
\end{equation*}

\subsection{Joint probability density functions}
\index{probability!probability density functions}

Probability density functions can also include more than one variable. Let $x$
and $y$ be random variables. The joint probability density function $p(x, y)$
defines the probability $p(x, y) \,dx \,dy$, so that $x$ and $y$ are in the
intervals $x \in [x, x + dx], y \in [y, y + dy]$. In other words, the
probability is the volume under a region of the PDF manifold (see figure
\ref{fig:joint_pdf} for an example of a joint PDF).
\begin{svg}{build/\chapterpath/joint_pdf}
  \caption{Joint probability density function}
  \label{fig:joint_pdf}
\end{svg}

Joint probability density functions also require that no probabilities are
negative and that the sum of all probabilities is $1$.
\begin{equation*}
  p(x, y) \geq 0, \int_{-\infty}^\infty \int_{-\infty}^{\infty} p(x, y) \,dx
    \,dy = 1
\end{equation*}

The expected values for joint PDFs are as follows.
\begin{align*}
  E[x] &= \int_{-\infty}^\infty \int_{-\infty}^{\infty} x p(x, y) \,dx \,dy \\
  E[y] &= \int_{-\infty}^\infty \int_{-\infty}^{\infty} y p(x, y) \,dx \,dy \\
  E[f(x, y)] &= \int_{-\infty}^\infty \int_{-\infty}^{\infty} f(x, y) p(x, y)
    \,dx \,dy
\end{align*}

The variance of a joint PDF measures how a variable correlates with itself
(we'll cover variances with respect to other variables shortly).
\begin{align*}
  \var(x) &= \Sigma_{xx} = E[(x - \overline{x})^2] =
    \int_{-\infty}^\infty \int_{-\infty}^\infty (x - \overline{x})^2 \,p(x, y)
    \,dx \,dy \\
  \var(y) &= \Sigma_{yy} = E[(y - \overline{y})^2] =
    \int_{-\infty}^\infty \int_{-\infty}^\infty (y - \overline{y})^2 \,p(x, y)
    \,dx \,dy \\
\end{align*}

\subsection{Covariance}
\index{probability!covariance}

A covariance is a measurement of how a variable correlates with another. If they
vary in the same direction, the covariance increases. If they vary in opposite
directions, the covariance decreases.
\begin{equation*}
  \cov(x, y) = \Sigma_{xy} = E[(x - \overline{x})(y - \overline{y})] =
    \int_{-\infty}^\infty \int_{-\infty}^\infty (x - \overline{x})
    (y - \overline{y}) \,p(x, y) \,dx \,dy \\
\end{equation*}

\subsection{Correlation}

Two random variables are correlated if the result of one random variable affects
the result of another. Correlation is defined as
\begin{equation*}
  \rho(x, y) = \frac{\Sigma_{xy}}{\sqrt{\Sigma_{xx}\Sigma_{yy}}}, |\rho(x, y)|
    \leq 1
\end{equation*}

So two variable's correlation is defined as their covariance over the geometric
mean of their variances. Uncorrelated sources have a covariance of zero.

\subsection{Independence}

Two random variables are independent if the following relation is true.
\begin{equation*}
  p(x, y) = p(x) \,p(y)
\end{equation*}

This means that the values of $x$ do not correlate with the values of $y$. That
is, the outcome of one random variable does not affect another's outcome. If we
assume independence,
\begin{align*}
  E[xy] &= \int_{-\infty}^\infty \int_{-\infty}^\infty xy \,p(x, y) \,dx \,dy \\
  E[xy] &= \int_{-\infty}^\infty \int_{-\infty}^\infty xy \,p(x) \,p(y) \,dx
    \,dy \\
  E[xy] &= \int_{-\infty}^\infty x \,p(x) \,dx \int_{-\infty}^\infty y \,p(y)
    \,dy \\
  E[xy] &= E[x]E[y] \\
  E[xy] &= \overline{x}\,\overline{y}
\end{align*}
\begin{align*}
  \cov(x, y) &= E[(x - \overline{x})(y - \overline{y})] \\
  \cov(x, y) &= E[(x - \overline{x})]E[(y - \overline{y})] \\
  \cov(x, y) &= 0 \cdot 0 \\
\end{align*}

Therefore, the covariance $\Sigma_{xy}$ is zero, as expected. Furthermore,
$\rho(x, y) = 0$, which means they are uncorrelated.

\subsection{Marginal probability density functions}
\index{probability!marginal probability density functions}

Given two random variables $x$ and $y$ whose joint distribution is known, the
marginal PDF $p(x)$ expresses the probability of $x$ averaged over information
about $y$. In other words, it's the PDF of $x$ when $y$ is unknown. This is
calculated by integrating the joint PDF over $y$.
\begin{equation*}
  p(x) = \int_{-\infty}^\infty p(x, y) \,dy
\end{equation*}

\subsection{Conditional probability density functions}
\index{probability!conditional probability density functions}

Let us assume that we know the joint PDF $p(x, y)$ and the exact value for $y$.
The conditional PDF gives the probability of $x$ in the interval $[x, x + dx]$
for the given value $y$. The probability of $x$ given $y$ is denoted by
$p(x|y)$.

If $p(x, y)$ is known, then we also know $p(x, y = y^\ast)$. However, note that
the latter is not the conditional density $p(x|y^\ast)$, instead
\begin{align*}
  C(y^\ast) &= \int_{-\infty}^\infty p(x, y = y^\ast) \,dx \\
  p(x|y^\ast) &= \frac{1}{C(y^\ast)} p(x, y = y^\ast)
\end{align*}

The scale factor $\frac{1}{C(y^\ast)}$ is used to scale the area under the PDF
to $1$.

\subsection{Bayes's rule}
\index{probability!Bayes's rule}

Bayes's rule is used to determine the probability of an event based on prior
knowledge of conditions related to the event.
\begin{equation*}
  p(x, y) = p(x|y) \,p(y) = p(y|x) \,p(x)
\end{equation*}

If $x$ and $y$ are independent, then $p(x|y) = p(x)$, $p(y|x) = p(y)$, and
$p(x, y) = p(x) \,p(y)$.

\subsection{Conditional expectation}
\index{probability!conditional expectation}

The concept of expectation can also be applied to conditional PDFs. This allows
us to determine what the mean of a variable is given prior knowledge of other
variables.
\begin{align*}
  E[x|y] &= \int_{-\infty}^\infty x \,p(x|y) \,dx = f(y), E[x|y] \neq E[x] \\
  E[y|x] &= \int_{-\infty}^\infty y \,p(y|x) \,dy = f(x), E[y|x] \neq E[y]
\end{align*}

\subsection{Conditional variances}
\index{probability!conditional variances}
\begin{align*}
  \var(x|y) &= E[(x - E[x|y])^2|y] \\
  \var(x|y) &= \int_{-\infty}^\infty (x - E[x|y])^2 \,p(x|y) \,dx
\end{align*}

\subsection{Random vectors}

Now we will extend the probability concepts discussed so far to vectors where
each element has a PDF.
\begin{equation*}
  \mat{x} = \begin{bmatrix}
    x_1 \\
    \vdots \\
    x_n
  \end{bmatrix}
\end{equation*}

The elements of $\mat{x}$ are scalar variables jointly distributed with a joint
density $p(x_1, \ldots, x_n)$. The expectation is
\begin{align*}
  E[\mat{x}] &= \matbar{x} = \int_{-\infty}^\infty \mat{x} \,p(\mat{x})
    \,d\mat{x} \\
  E[\mat{x}] &= \begin{bmatrix}
    E[x_1] \\
    \vdots \\
    E[x_n]
  \end{bmatrix} \\
  E[x_i] &= \int_{-\infty}^\infty \ldots \int_{-\infty}^\infty x_i
    \,p(x_1, \ldots, x_n) \,dx_1 \ldots dx_n \\
  E[f(\mat{x})] &= \int_{-\infty}^\infty f(\mat{x}) \,p(\mat{x}) \,d\mat{x}
\end{align*}

\subsection{Covariance matrix}
\index{probability!covariance matrix}

The covariance matrix for a random vector $\mat{x} \in \mathbb{R}^n$ is
\begin{align*}
  \mat{\Sigma} &= \cov(\mat{x}, \mat{x}) = E[(\mat{x} - \matbar{x})
    (\mat{x} - \matbar{x})\T] \\
  \mat{\Sigma} &= \begin{bmatrix}
    \cov(x_1, x_1) & \ldots & \cov(x_1, x_n) \\
    \vdots         & \ddots & \vdots \\
    \cov(x_n, x_1) & \ldots & \cov(x_n, x_n) \\
  \end{bmatrix}
\end{align*}

This $n \times n$ matrix is symmetric and positive semidefinite. A positive
semidefinite matrix satisfies the relation that for any
$\mat{v} \in \mathbb{R}^n$ for which $\mat{v} \neq 0$,
$\mat{v}\T \mat{\Sigma} \mat{v} \geq 0$. In other words, the eigenvalues of
$\mat{\Sigma}$ are all greater than or equal to zero.

\subsection{Relations for independent random vectors}

First, independent vectors imply linearity from
$p(\mat{x}, \mat{y}) = p(\mat{x}) \,p(\mat{y})$.
\begin{align*}
  E[\mat{A}\mat{x} + \mat{B}\mat{y}] &= \mat{A}E[\mat{x}] + \mat{B}E[\mat{y}] \\
  E[\mat{A}\mat{x} + \mat{B}\mat{y}] &= \mat{A}\matbar{x} + \mat{B}\matbar{y}
\end{align*}

Second, independent vectors being uncorrelated means their covariance is zero.
\begin{align}
  \mat{\Sigma}_{\mat{x}\mat{y}} &= \cov(\mat{x}, \mat{y}) \nonumber \\
  \mat{\Sigma}_{\mat{x}\mat{y}} &= E[(\mat{x} - \matbar{x})
    (\mat{y} - \matbar{y})\T] \nonumber \\
  \mat{\Sigma}_{\mat{x}\mat{y}} &= E[\mat{x}\mat{y}\T] -
    E[\mat{x}\matbar{y}\T] - E[\matbar{x}\mat{y}\T] +
    E[\matbar{x}\matbar{y}\T] \nonumber \\
  \mat{\Sigma}_{\mat{x}\mat{y}} &= E[\mat{x}\mat{y}\T] -
    E[\mat{x}]\matbar{y}\T - \matbar{x}E[\mat{y}\T] +
    \matbar{x}\matbar{y}\T \nonumber \\
  \mat{\Sigma}_{\mat{x}\mat{y}} &= E[\mat{x}\mat{y}\T] -
    \matbar{x}\matbar{y}\T - \matbar{x}\matbar{y}\T +
    \matbar{x}\matbar{y}\T \nonumber \\
  \mat{\Sigma}_{\mat{x}\mat{y}} &= E[\mat{x}\mat{y}\T] -
    \matbar{x}\matbar{y}\T \label{eq:prb_sigma}
\end{align}

Now, compute $E[\mat{x}\mat{y}\T]$.
\begin{equation*}
  E[\mat{x}\mat{y}\T] = \int_X \int_Y \mat{x}\mat{y}\T \,p(\mat{x})
    \,p(\mat{y}) \,d\mat{x} \,d\mat{y}\T
\end{equation*}

Factor out constants from the inner integral. This includes variables which are
held constant for each inner integral evaluation.
\begin{equation*}
  E[\mat{x}\mat{y}\T] = \int_X p(\mat{x}) \,\mat{x} \,d\mat{x}
    \int_Y p(\mat{y}) \,\mat{y}\T \,d\mat{y}\T
\end{equation*}

Each of these integrals is just the expected value of their respective
integration variable.
\begin{align}
  E[\mat{x}\mat{y}\T] &= \matbar{x}\matbar{y}\T \label{eq:prb_exyt}
\end{align}

Substitute equation \eqref{eq:prb_exyt} into equation \eqref{eq:prb_sigma}.
\begin{align*}
  \mat{\Sigma}_{\mat{x}\mat{y}} &= (\matbar{x}\matbar{y}\T) -
    \matbar{x}\matbar{y}\T \\
  \mat{\Sigma}_{\mat{x}\mat{y}} &= 0
\end{align*}

Using these results, we can compute the covariance of
$\mat{z} = \mat{A}\mat{x} + \mat{B}\mat{y}$.
\begin{align*}
  \Sigma_z &= \cov(\mat{z}, \mat{z}) \\
  \Sigma_z &= E[(\mat{z} - \matbar{z})(\mat{z} - \matbar{z})\T] \\
  \Sigma_z &= E[(\mat{A}\mat{x} + \mat{B}\mat{y} - \mat{A}\matbar{x} -
    \mat{B}\matbar{y})(\mat{A}\mat{x} + \mat{B}\mat{y} -
    \mat{A}\matbar{x} - \mat{B}\matbar{y})\T] \\
  \Sigma_z &= E[(\mat{A}(\mat{x} - \matbar{x}) +
    \mat{B}(\mat{y} - \matbar{y}))
    (\mat{A}(\mat{x} - \matbar{x}) +
     \mat{B}(\mat{y} - \matbar{y}))\T] \\
  \Sigma_z &= E[(\mat{A}(\mat{x} - \matbar{x}) +
    \mat{B}(\mat{y} - \matbar{y}))
    ((\mat{x} - \matbar{x})\T\mat{A}\T +
     (\mat{y} - \matbar{y})\T\mat{B}\T)] \\
  \Sigma_z &= E[
    \mat{A}(\mat{x} - \matbar{x})(\mat{x} - \matbar{x})\T\mat{A}\T +
    \mat{A}(\mat{x} - \matbar{x})(\mat{y} - \matbar{y})\T\mat{B}\T \\
    &\qquad + \mat{B}(\mat{y} - \matbar{y})(\mat{x} - \matbar{x})\T\mat{A}\T +
    \mat{B}(\mat{y} - \matbar{y})(\mat{y} - \matbar{y})\T\mat{B}\T]
  \intertext{Since $\mat{x}$ and $\mat{y}$ are independent, $\Sigma_{xy} = 0$
    and the cross terms cancel out.}
  \Sigma_z &= E[
    \mat{A}(\mat{x} - \matbar{x})(\mat{x} - \matbar{x})\T\mat{A}\T + 0 + 0 +
    \mat{B}(\mat{y} - \matbar{y})(\mat{y} - \matbar{y})\T\mat{B}\T] \\
  \Sigma_z &=
    E[\mat{A}(\mat{x} - \matbar{x})(\mat{x} - \matbar{x})\T\mat{A}\T] +
    E[\mat{B}(\mat{y} - \matbar{y})(\mat{y} - \matbar{y})\T\mat{B}\T] \\
  \Sigma_z &=
    \mat{A}E[(\mat{x} - \matbar{x})(\mat{x} - \matbar{x})\T]\mat{A}\T +
    \mat{B}E[(\mat{y} - \matbar{y})(\mat{y} - \matbar{y})\T]\mat{B}\T
  \intertext{Recall that $\Sigma_x = \cov(\mat{x}, \mat{x})$ and
    $\Sigma_y = \cov(\mat{y}, \mat{y})$.}
  \Sigma_z &= \mat{A}\Sigma_x\mat{A}\T + \mat{B}\Sigma_y\mat{B}\T
\end{align*}

\subsection{Gaussian random variables}

A Gaussian random variable has the following properties:
\begin{align*}
  E[x] &= \overline{x} \\
  \var(x) &= \sigma^2 \\
  p(x) &= \frac{1}{\sqrt{2\pi\sigma^2}}
    e^{-\frac{(x - \overline{x})^2}{2\sigma^2}}
\end{align*}

While we could use any random variable to represent a random process, we use the
Gaussian random variable a lot in probability theory due to the central limit
theorem.
\begin{definition}[Central limit theorem]
  When independent random variables are added, their properly normalized sum
  tends toward a normal distribution (a Gaussian distribution or ``bell curve").
\end{definition}
\index{probability!central limit theorem}

This is the case even if the original variables themselves are not normally
distributed. The theorem is a key concept in probability theory because it
implies that probabilistic and statistical methods that work for normal
distributions can be applicable to many problems involving other types of
distributions.

For example, suppose that a sample is obtained containing a large number of
independent observations, and that the arithmetic mean of the observed values
is computed. The central limit theorem says that the computed values of the
mean will tend toward being distributed according to a normal distribution.
\begin{bookfigure}
  \begin{bookminifig}{2}
    \qrcode{https://youtu.be/zeJD6dqJ5lo} \\
    ``But what is the Central Limit Theorem?" (31 minutes) \\
    \footnotesize 3Blue1Brown \\
    \url{https://youtu.be/zeJD6dqJ5lo}
  \end{bookminifig}
  \begin{bookminifig}{2}
    \qrcode{https://youtu.be/d_qvLDhkg00} \\
    ``A pretty reason why Gaussian + Gaussian = Gaussian" (13 minutes) \\
    \footnotesize 3Blue1Brown \\
    \url{https://youtu.be/d_qvLDhkg00}
  \end{bookminifig}
\end{bookfigure}

\input{\chapterpath/linear-stochastic-systems}
\section{Two-sensor problem}
\index{stochastic!two-sensor problem}

\subsection{Two noisy (independent) observations}

Variable $z_1$ is the noisy measurement of the variable $x$. The value of the
noisy measurement is the random variable defined by the probability density
function $p(z_1|x)$.

Variable $z_2$ is the noisy measurement of the same variable x and independent
of $z_1$. If we know the property of our sensor (measurement noise), then the
probability density function of the measurement is $p(z_2|x)$.

We are interested in the probability density function of $x$ given the
measurements $z_1$ and $z_2$.
\begin{equation*}
  p(x|z_1, z_2) = \frac{p(x, z_1, z_2)}{p(z_1, z_2)}
                = \frac{p(z_1, z_2|x)p(x)}{p(z_1, z_2)}
                = \frac{p(z_1|x) p(z_2|x)}{p(z_1, z_2)}
\end{equation*}

where $p(z_1, z_2) = \int_X p(z_1|x) p(z_2|x) \,dx$, but $z_1$ and $z_2$ are
given and we can write
\begin{equation*}
  p(x|z_1, z_2) = \frac{1}{C} p(z_1|x) p(z_2|x) \quad \text{or} \quad
    p(x|z_1, z_2) \sim p(z_1|x) p(z_2|x)
\end{equation*}

where $C$ is a normalizing constant providing that
$\int_X p(x|z_1, z_2) \,dx = 1$. The probability density $p(x|z_1, z_2)$
summarizes our complete knowledge about $x$.

\subsection{Single noisy observations}

Variable $z_1$ is the noisy measurement of the variable $x$. The value of the
noisy measurement is the random variable defined by the probability density
function $p(z_1|x)$.

The probability density function of $x$ given the measurement $z_1$ is
\begin{equation*}
  p(x|z_1) = \frac{p(x, z_1)}{p(z_1)} = \frac{p(z_1|x) p(x)}{p(z_1)}
\end{equation*}

where $p(z_1) = \int_X p(z_1|x) \,dx$, but $z_1$ is given so we can write
\begin{equation*}
  p(x|z_1) = \frac{1}{C} p(z_1|x) p(x) \quad \text{or} \quad
    p(x|z_1) \sim p(z_1|x) p(x)
\end{equation*}

The probability density $p(x|z_1)$ summarizes our complete knowledge about $x$.
\begin{remark}
  Even in the single measurement case, the estimation of the variable $x$ is a
  data/information fusion problem. We combine \textit{prior probability} with
  the probability resulting from the measurement.
\end{remark}

\subsection{Single noisy observations: a Gaussian case}
\begin{equation*}
  p(z_1|x) = \frac{1}{\sigma \sqrt{2\pi}} e^{-\frac{1}{2}
    \left(\frac{z_1 - x}{\sigma}\right)^2}
  \quad \text{and} \quad
  p(x) = \frac{1}{\sigma_0 \sqrt{2\pi}} e^{-\frac{1}{2}
    \left(\frac{x - x_0}{\sigma_0}\right)^2}
\end{equation*}

$z_1$, $x_0$, $\sigma^2$, and $\sigma_0^2$ are given.
\begin{align*}
  p(x|z_1) &= \frac{p(x, z_1)}{p(z_1)} \\
  p(x|z_1) &= \frac{p(z_1|x) p(x)}{p(z_1)} \\
  p(x|z_1) &= \frac{1}{C} p(z_1|x) p(x) \\
  p(x|z_1) &= \frac{1}{C}
    \frac{1}{\sigma \sqrt{2\pi}} e^{-\frac{1}{2}
      \left(\frac{z_1 - x}{\sigma}\right)^2}
    \frac{1}{\sigma_0 \sqrt{2\pi}} e^{-\frac{1}{2}
      \left(\frac{x - x_0}{\sigma_0}\right)^2}
  \intertext{Absorb the leading coefficients of the two probability
    distributions into a new constant $C_1$.}
  p(x|z_1) &= \frac{1}{C_1}
    e^{-\frac{1}{2} \left(\frac{z_1 - x}{\sigma}\right)^2}
    e^{-\frac{1}{2} \left(\frac{x - x_0}{\sigma_0}\right)^2}
  \intertext{Combine the exponents.}
  p(x|z_1) &= \frac{1}{C_1} e^{-\frac{1}{2} \left(
      \frac{(z_1 - x)^2}{\sigma^2} + \frac{(x - x_0)^2}{\sigma_0^2}
    \right)}
  \intertext{Expand the exponent into separate terms.}
  p(x|z_1) &= \frac{1}{C_1} e^{-\frac{1}{2} \left(
      \frac{z_1^2}{\sigma^2} - \frac{2z_1 x}{\sigma^2} + \frac{x^2}{\sigma^2} +
      \frac{x^2}{\sigma_0^2} - \frac{2xx_0}{\sigma_0^2} + \frac{x_0^2}{\sigma_0^2}
    \right)}
  \intertext{Multiply in
    $\frac{\sigma^2}{\sigma^2}$ or $\frac{\sigma_0^2}{\sigma_0^2}$ as
    appropriate to give all terms a denominator of $\sigma^2 \sigma_0^2$.}
  p(x|z_1) &= \frac{1}{C_1} e^{-\frac{1}{2} \left(
      \frac{\sigma_0^2 z_1^2}{\sigma^2 \sigma_0^2}
      - 2\frac{\sigma_0^2 z_1}{\sigma^2 \sigma_0^2} x
      + \frac{\sigma_0^2}{\sigma^2 \sigma_0^2} x^2
      + \frac{\sigma^2}{\sigma^2 \sigma_0^2} x^2
      - 2\frac{\sigma^2 x_0}{\sigma^2 \sigma_0^2} x
      + \frac{\sigma^2 x_0^2}{\sigma^2 \sigma_0^2}
    \right)}
  \intertext{Reorder terms in the exponent.}
  p(x|z_1) &= \frac{1}{C_1} e^{-\frac{1}{2} \left(
      \frac{\sigma_0^2}{\sigma^2 \sigma_0^2} x^2
      + \frac{\sigma^2}{\sigma^2 \sigma_0^2} x^2
      - 2\frac{\sigma_0^2 z_1}{\sigma^2 \sigma_0^2} x
      - 2\frac{\sigma^2 x_0}{\sigma^2 \sigma_0^2} x
      + \frac{\sigma_0^2 z_1^2}{\sigma^2 \sigma_0^2}
      + \frac{\sigma^2 x_0^2}{\sigma^2 \sigma_0^2}
    \right)}
  \intertext{Combine like terms.}
  p(x|z_1) &= \frac{1}{C_1} e^{-\frac{1}{2} \left(
      \frac{\sigma_0^2 + \sigma^2}{\sigma^2\sigma_0^2} x^2 -
      2\frac{\sigma_0^2 z_1 + \sigma^2 x_0}{\sigma^2 \sigma_0^2} x +
      \frac{\sigma_0^2 z_1^2 + \sigma^2 x_0^2}{\sigma^2 \sigma_0^2}
    \right)}
  \intertext{Factor out $\frac{\sigma_0^2 + \sigma^2}{\sigma^2 \sigma_0^2}$.}
  p(x|z_1) &= \frac{1}{C_1} e^{-\frac{1}{2}
    \frac{\sigma_0^2 + \sigma^2}{\sigma^2 \sigma_0^2} \left(
      x^2 - 2\frac{\sigma_0^2 z_1 + \sigma^2 x_0}{\sigma_0^2 + \sigma^2} x +
      \frac{\sigma_0^2 z_1^2 + \sigma^2 x_0^2}{\sigma_0^2 + \sigma^2}
    \right)} \\
  p(x|z_1) &= \frac{1}{C_1} e^{-\frac{1}{2}
    \frac{\sigma_0^2 + \sigma^2}{\sigma^2 \sigma_0^2} \left(
      x^2 - 2\left(
        \frac{\sigma_0^2}{\sigma_0^2 + \sigma^2} z_1 +
        \frac{\sigma^2}{\sigma_0^2 + \sigma^2} x_0
      \right) x +
      \frac{\sigma_0^2 z_1^2 + \sigma^2 x_0^2}{\sigma_0^2 + \sigma^2}
    \right)}
  \intertext{$\frac{\sigma_0^2}{\sigma^2 + \sigma_0^2} z_1 + \frac{\sigma^2}{\sigma_0^2 + \sigma^2} x_0$
    is the mean of the combined probability distribution, which we'll denote as
    $\mu$.}
  p(x|z_1) &= \frac{1}{C_1} e^{-\frac{1}{2}
    \frac{\sigma^2 + \sigma_0^2}{\sigma^2 \sigma_0^2} \left(
      x^2 - 2\mu x +
      \frac{\sigma_0^2 z_1^2 + \sigma^2 x_0^2}{\sigma_0^2 + \sigma^2}
    \right)}
  \intertext{Add in $\mu^2 - \mu^2$ to perform some factoring.}
  p(x|z_1) &= \frac{1}{C_1} e^{-\frac{1}{2}
    \frac{\sigma_0^2 + \sigma^2}{\sigma^2 \sigma_0^2} \left(
      x^2 - 2\mu x + \mu^2 - \mu^2 +
      \frac{\sigma_0^2 z_1^2 + \sigma^2 x_0^2}{\sigma_0^2 + \sigma^2}
    \right)} \\
  p(x|z_1) &= \frac{1}{C_1} e^{-\frac{1}{2}
    \frac{\sigma_0^2 + \sigma^2}{\sigma^2 \sigma_0^2} \left(
      (x - \mu)^2 - \mu^2 +
      \frac{\sigma_0^2 z_1^2 + \sigma^2 x_0^2}{\sigma_0^2 + \sigma^2}
    \right)}
  \intertext{Pull out all constant terms in the exponent and combine them with
    $C_1$ to make a new constant $C_2$. We're basically doing
    $c_1 e^{x + a} \rightarrow c_1 e^a e^x \rightarrow c_2 e^x$.}
  p(x|z_1) &= \frac{1}{C_2} e^{-\frac{1}{2}
    \frac{\sigma_0^2 + \sigma^2}{\sigma^2 \sigma_0^2} (x - \mu)^2}
\end{align*}

This means that if we're given an initial estimate $x_0$ and a measurement $z_1$
with associated means and variances represented by Gaussian distributions, this
information can be combined into a third Gaussian distribution with its own mean
value and variance. The expected value of $x$ given $z_1$ is
\begin{equation}
  E[x|z_1] = \mu = \frac{\sigma_0^2}{\sigma_0^2 + \sigma^2}z_1 +
    \frac{\sigma^2}{\sigma_0^2 + \sigma^2}x_0
\end{equation}

The variance of $x$ given $z_1$ is
\begin{equation}
  E[(x - \mu)^2|z_1] = \frac{\sigma^2 \sigma_0^2}{\sigma_0^2 + \sigma^2}
\end{equation}

The expected value, which is also the maximum likelihood value, is the linear
combination of the prior expected (maximum likelihood) value and the
measurement. The expected value is a reasonable estimator of $x$.
\begin{align}
  \hat{x} &= E[x|z_1] = \frac{\sigma_0^2}{\sigma_0^2 + \sigma^2}z_1 +
    \frac{\sigma^2}{\sigma_0^2 + \sigma^2}x_0 \\
  \hat{x} &= w_1 z_1 + w_2 x_0 \nonumber
\end{align}

Note that the weights $w_1$ and $w_2$ sum to $1$. When the prior (i.e., prior
knowledge of \gls{state}) is uninformative (a large variance),
\begin{align}
  w_1 &= \lim_{\sigma_0^2 \to \infty} \frac{\sigma_0^2}{\sigma_0^2 + \sigma^2} = 1 \\
  w_2 &= \lim_{\sigma_0^2 \to \infty} \frac{\sigma^2}{\sigma_0^2 + \sigma^2} = 0
\end{align}

and $\hat{x} = z_1$. That is, the weight is on the observations and the estimate
is equal to the measurement.

Let us assume we have a \gls{model} providing an almost exact prior for $x$. In
that case, $\sigma_0^2$ approaches 0 and
\begin{align}
  w_1 &= \lim_{\sigma_0^2 \to 0} \frac{\sigma_0^2}{\sigma_0^2 + \sigma^2} = 0 \\
  w_2 &= \lim_{\sigma_0^2 \to 0} \frac{\sigma^2}{\sigma_0^2 + \sigma^2} = 1
\end{align}

The Kalman filter uses this optimal fusion as the basis for its operation.

\section{Kalman filter}

So far, we've derived equations for updating the expected value and state
covariance without measurements and how to incorporate measurements into an
initial \gls{state} optimally. Now, we'll combine these concepts to produce an
estimator which minimizes the error covariance for linear \glspl{system}.

\subsection{Derivations}
\index{state-space observers!Kalman filter!derivations}

Given the \textit{a posteriori} update equation
$\hat{\mat{x}}_{k+1}^+ = \hat{\mat{x}}_{k+1}^- + \mat{K}_{k+1}(\mat{y}_{k+1} -
\hat{\mat{y}}_{k+1})$, we want to find the value of $\mat{K}_{k+1}$ that
minimizes the \textit{a posteriori} estimate covariance (the error covariance)
because this minimizes the estimation error.

\subsubsection{\textit{a posteriori} estimate covariance update equation}

The following is the definition of the \textit{a posteriori} estimate covariance
matrix.
\begin{equation*}
  \mat{P}_{k+1}^+ = \cov(\mat{x}_{k+1} - \hat{\mat{x}}_{k+1}^+)
\end{equation*}

Substitute in the \textit{a posteriori} update equation and expand the
measurement equations.
\begin{align*}
  \mat{P}_{k+1}^+ &= \cov(\mat{x}_{k+1} - (\hat{\mat{x}}_{k+1}^- +
    \mat{K}_{k+1}(\mat{y}_{k+1} - \hat{\mat{y}}_{k+1}))) \\
  \mat{P}_{k+1}^+ &= \cov(\mat{x}_{k+1} - \hat{\mat{x}}_{k+1}^- -
    \mat{K}_{k+1}(\mat{y}_{k+1} - \hat{\mat{y}}_{k+1})) \\
  \mat{P}_{k+1}^+ &= \cov(\mat{x}_{k+1} - \hat{\mat{x}}_{k+1}^- -
    \mat{K}_{k+1}( \\
      &\qquad (\mat{C}_{k+1} \mat{x}_{k+1} + \mat{D}_{k+1} \mat{u}_{k+1} +
        \mat{v}_{k+1}) \\
      &\qquad - (\mat{C}_{k+1} \hat{\mat{x}}_{k+1}^- +
        \mat{D}_{k+1}\mat{u}_{k+1}))) \\
  \mat{P}_{k+1}^+ &= \cov(\mat{x}_{k+1} - \hat{\mat{x}}_{k+1}^- -
    \mat{K}_{k+1}( \\
      &\qquad \mat{C}_{k+1} \mat{x}_{k+1} + \mat{D}_{k+1} \mat{u}_{k+1} +
        \mat{v}_{k+1} \\
      &\qquad - \mat{C}_{k+1} \hat{\mat{x}}_{k+1}^- -
        \mat{D}_{k+1}\mat{u}_{k+1}))
  \intertext{Reorder terms.}
  \mat{P}_{k+1}^+ &= \cov(\mat{x}_{k+1} - \hat{\mat{x}}_{k+1}^- -
    \mat{K}_{k+1}( \\
      &\qquad\mat{C}_{k+1} \mat{x}_{k+1} -
        \mat{C}_{k+1} \hat{\mat{x}}_{k+1}^- \\
      &\qquad + \mat{D}_{k+1} \mat{u}_{k+1} - \mat{D}_{k+1}\mat{u}_{k+1} +
        \mat{v}_{k+1}))
  \intertext{The $\mat{D}_{k+1}\mat{u}_{k+1}$ terms cancel.}
  \mat{P}_{k+1}^+ &= \cov(\mat{x}_{k+1} - \hat{\mat{x}}_{k+1}^- -
    \mat{K}_{k+1}(\mat{C}_{k+1}\mat{x}_{k+1} -
    \mat{C}_{k+1} \hat{\mat{x}}_{k+1}^- + \mat{v}_{k+1}))
  \intertext{Distribute $\mat{K}_{k+1}$ to $\mat{v}_{k+1}$.}
  \mat{P}_{k+1}^+ &= \cov(\mat{x}_{k+1} - \hat{\mat{x}}_{k+1}^- -
    \mat{K}_{k+1}(\mat{C}_{k+1}\mat{x}_{k+1} -
    \mat{C}_{k+1} \hat{\mat{x}}_{k+1}^-) - \mat{K}_{k+1}\mat{v}_{k+1})
  \intertext{Factor out $\mat{C}_{k+1}$.}
  \mat{P}_{k+1}^+ &= \cov(\mat{x}_{k+1} - \hat{\mat{x}}_{k+1}^- -
    \mat{K}_{k+1}\mat{C}_{k+1}(\mat{x}_{k+1} - \hat{\mat{x}}_{k+1}^-) -
    \mat{K}_{k+1}\mat{v}_{k+1})
  \intertext{Factor out $\mat{x}_{k+1} - \hat{\mat{x}}_{k+1}^-$ to the right.}
  \mat{P}_{k+1}^+ &= \cov((\mat{I} - \mat{K}_{k+1}\mat{C}_{k+1})
    (\mat{x}_{k+1} - \hat{\mat{x}}_{k+1}^-) - \mat{K}_{k+1}\mat{v}_{k+1})
  \intertext{Covariance is a linear operator, so it can be applied to each term
    separately. Covariance squares terms internally, so the negative sign on
    $\mat{K}_{k+1}\mat{v}_{k+1}$ is removed.}
  \mat{P}_{k+1}^+ &= \cov((\mat{I} - \mat{K}_{k+1}\mat{C}_{k+1})
    (\mat{x}_{k+1} - \hat{\mat{x}}_{k+1}^-)) + \cov(\mat{K}_{k+1}\mat{v}_{k+1})
  \intertext{Now just evaluate the covariances.}
  \mat{P}_{k+1}^+ &= (\mat{I} - \mat{K}_{k+1}\mat{C}_{k+1})
    \cov(\mat{x}_{k+1} - \hat{\mat{x}}_{k+1}^-)
    (\mat{I} - \mat{K}_{k+1}\mat{C}_{k+1})\T \\
      &\qquad + \mat{K}_{k+1}cov(\mat{v}_{k+1})\mat{K}_{k+1}\T \\
  \mat{P}_{k+1}^+ &= (\mat{I} - \mat{K}_{k+1}\mat{C}_{k+1})\mat{P}_{k+1}^-
    (\mat{I} - \mat{K}_{k+1}\mat{C}_{k+1})\T + \mat{K}_{k+1}\mat{R}_{k+1}
    \mat{K}_{k+1}\T
\end{align*}

\subsubsection{Finding the optimal Kalman gain}

The error in the \textit{a posteriori} \gls{state} estimation is
$\mat{x}_{k+1} - \hat{\mat{x}}_{k+1}^+$. We want to minimize the expected value
of the square of the magnitude of this vector, which can be written as
$E[(\mat{x}_{k+1} - \hat{\mat{x}}_{k+1}^+)(\mat{x}_{k+1} - \hat{\mat{x}}_{k+1}^+)\T]$.
By definition, this expected value is the \textit{a posteriori} error covariance
$\mat{P}_{k+1}^+$. Remember that the eigenvectors of a matrix are the
fundamental transformation directions of that matrix, and the eigenvalues are
the magnitudes of those transformations. By minimizing the eigenvalues, we
minimize the error variance (the uncertainty in the state estimate) for each
state.

$\mat{P}_{k+1}^+$ is positive definite, so we know all the eigenvalues are
positive. Therefore, a reasonable quantity to minimize with our choice of Kalman
gain is the sum of the eigenvalues. We don't have direct access to the
eigenvalues, but we can use the fact that the sum of the eigenvalues is equal to
the trace of $\mat{P}_{k+1}^+$ and minimize that instead.\footnote{By
definition, the characteristic polynomial of an $n \times n$ matrix $\mat{P}$ is
given by
\begin{equation*}
  p(t) = \det(t\mat{I} - \mat{P}) = t^n - \tr(\mat{P}) t^{n-1} + \cdots +
    (-1)^n \det(\mat{P})
\end{equation*}

as well as $p(t) = (t - \lambda_1) \ldots (t - \lambda_n)$ where
$\lambda_1, \ldots, \lambda_n$ are the eigenvalues of $\mat{P}$. The coefficient
for $t^{n-1}$ in the second polynomial's expansion is
$-(\lambda_1 + \cdots + \lambda_n)$. Therefore, by matching coefficients for
$t^{n-1}$, we get $\tr(\mat{P}) = \lambda_1 + \cdots + \lambda_n$.}

We'll start with the equation for $\mat{P}_{k+1}^+$.
\begin{align*}
  \mat{P}_{k+1}^+ &= (\mat{I} - \mat{K}_{k+1}\mat{C}_{k+1})\mat{P}_{k+1}^-
    (\mat{I} - \mat{K}_{k+1}\mat{C}_{k+1})\T + \mat{K}_{k+1}\mat{R}_{k+1}
    \mat{K}_{k+1}\T
  \intertext{We're going to expand the equation for $\mat{P}_{k+1}^+$ and
    collect terms. First, multiply in $\mat{P}_{k+1}^-$.}
  \mat{P}_{k+1}^+ &=
    (\mat{P}_{k+1}^- - \mat{K}_{k+1}\mat{C}_{k+1}\mat{P}_{k+1}^-)
    (\mat{I} - \mat{K}_{k+1}\mat{C}_{k+1})\T + \mat{K}_{k+1}\mat{R}_{k+1}
    \mat{K}_{k+1}\T
  \intertext{Tranpose each term in $\mat{I} - \mat{K}_{k+1}\mat{C}_{k+1}$.
    $\mat{I}$ is symmetric, so its transpose is dropped.}
  \mat{P}_{k+1}^+ &=
    (\mat{P}_{k+1}^- - \mat{K}_{k+1}\mat{C}_{k+1}\mat{P}_{k+1}^-)
    (\mat{I} - \mat{C}_{k+1}\T\mat{K}_{k+1}\T) +
    \mat{K}_{k+1}\mat{R}_{k+1} \mat{K}_{k+1}\T
  \intertext{Multiply in $\mat{I} - \mat{C}_{k+1}\T\mat{K}_{k+1}\T$.}
  \mat{P}_{k+1}^+ &=
    \mat{P}_{k+1}^-(\mat{I} - \mat{C}_{k+1}\T\mat{K}_{k+1}\T) -
    \mat{K}_{k+1}\mat{C}_{k+1}\mat{P}_{k+1}^-
    (\mat{I} - \mat{C}_{k+1}\T\mat{K}_{k+1}\T) \\
      &\qquad + \mat{K}_{k+1}\mat{R}_{k+1} \mat{K}_{k+1}\T
  \intertext{Expand terms.}
  \mat{P}_{k+1}^+ &=
    \mat{P}_{k+1}^- - \mat{P}_{k+1}^-\mat{C}_{k+1}\T\mat{K}_{k+1}\T -
    \mat{K}_{k+1}\mat{C}_{k+1}\mat{P}_{k+1}^- \\
      &\qquad + \mat{K}_{k+1}\mat{C}_{k+1}\mat{P}_{k+1}^-\mat{C}_{k+1}\T\mat{K}_{k+1}\T +
        \mat{K}_{k+1}\mat{R}_{k+1} \mat{K}_{k+1}\T
  \intertext{Factor out $\mat{K}_{k+1}$ and $\mat{K}_{k+1}\T$.}
  \mat{P}_{k+1}^+ &=
    \mat{P}_{k+1}^- - \mat{P}_{k+1}^-\mat{C}_{k+1}\T\mat{K}_{k+1}\T -
    \mat{K}_{k+1}\mat{C}_{k+1}\mat{P}_{k+1}^- \nonumber \\
      &\qquad + \mat{K}_{k+1}(\mat{C}_{k+1}\mat{P}_{k+1}^-\mat{C}_{k+1}\T +
        \mat{R}_{k+1})\mat{K}_{k+1}\T
\end{align*}

$\mat{C}_{k+1}\mat{P}_{k+1}^-\mat{C}_{k+1}\T + \mat{R}_{k+1}$ is the innovation
(measurement residual) covariance at timestep $k + 1$. We'll let this expression
equal $\mat{S}_{k+1}$. We won't need it in the final theorem, but it makes the
derivations after this point more concise.
\begin{equation}
  \mat{P}_{k+1}^+ =
    \mat{P}_{k+1}^- - \mat{P}_{k+1}^-\mat{C}_{k+1}\T\mat{K}_{k+1}\T -
    \mat{K}_{k+1}\mat{C}_{k+1}\mat{P}_{k+1}^- +
    \mat{K}_{k+1}\mat{S}_{k+1}\mat{K}_{k+1}\T \label{eq:post_p_update}
\end{equation}

Now take the trace.
\begin{align*}
  \tr(\mat{P}_{k+1}^+) &=
    \tr(\mat{P}_{k+1}^-) - \tr(\mat{P}_{k+1}^-\mat{C}_{k+1}\T\mat{K}_{k+1}\T) -
    \tr(\mat{K}_{k+1}\mat{C}_{k+1}\mat{P}_{k+1}^-) \\
    &\qquad + \tr(\mat{K}_{k+1}\mat{S}_{k+1}\mat{K}_{k+1}\T)
  \intertext{Transpose one of the terms twice.}
  \tr(\mat{P}_{k+1}^+) &= \tr(\mat{P}_{k+1}^-) -
    \tr((\mat{K}_{k+1}\mat{C}_{k+1}\mat{P}_{k+1}^{-\mathsf{T}})\T) -
    \tr(\mat{K}_{k+1}\mat{C}_{k+1}\mat{P}_{k+1}^-) \\
    &\qquad + \tr(\mat{K}_{k+1}\mat{S}_{k+1}\mat{K}_{k+1}\T)
  \intertext{$\mat{P}_{k+1}^-$ is symmetric, so we can drop the transpose.}
  \tr(\mat{P}_{k+1}^+) &= \tr(\mat{P}_{k+1}^-) -
    \tr((\mat{K}_{k+1}\mat{C}_{k+1}\mat{P}_{k+1}^-)\T) -
    \tr(\mat{K}_{k+1}\mat{C}_{k+1}\mat{P}_{k+1}^-) \\
    &\qquad + \tr(\mat{K}_{k+1}\mat{S}_{k+1}\mat{K}_{k+1}\T)
  \intertext{The trace of a matrix is equal to the trace of its transpose since
    the elements used in the trace are on the diagonal.}
  \tr(\mat{P}_{k+1}^+) &= \tr(\mat{P}_{k+1}^-) -
    \tr(\mat{K}_{k+1}\mat{C}_{k+1}\mat{P}_{k+1}^-) -
    \tr(\mat{K}_{k+1}\mat{C}_{k+1}\mat{P}_{k+1}^-) \\
    &\qquad + \tr(\mat{K}_{k+1}\mat{S}_{k+1}\mat{K}_{k+1}\T) \\
  \tr(\mat{P}_{k+1}^+) &= \tr(\mat{P}_{k+1}^-) -
    2\tr(\mat{K}_{k+1}\mat{C}_{k+1}\mat{P}_{k+1}^-) +
    \tr(\mat{K}_{k+1}\mat{S}_{k+1}\mat{K}_{k+1}\T)
\end{align*}

Given theorems \ref{thm:partial_tr_aba} and \ref{thm:partial_tr_ac}
\begin{theorem}
  \label{thm:partial_tr_aba}

  $\frac{\partial}{\partial\mat{A}}\tr(\mat{A}\mat{B}\mat{A}\T) =
    2\mat{A}\mat{B}$ where $\mat{B}$ is symmetric.
\end{theorem}
\begin{theorem}
  \label{thm:partial_tr_ac}

  $\frac{\partial}{\partial\mat{A}}\tr(\mat{A}\mat{C}) = \mat{C}\T$
\end{theorem}

find the minimum of the trace of $\mat{P}_{k+1}^+$ by taking the partial
derivative with respect to $\mat{K}$ and setting the result to $\mat{0}$.
\begin{align*}
  \frac{\partial\tr(\mat{P}_{k+1}^+)}{\partial\mat{K}} &=
    \mat{0} - 2(\mat{C}_{k+1}\mat{P}_{k+1}^-)\T + 2\mat{K}_{k+1}\mat{S}_{k+1} \\
  \frac{\partial\tr(\mat{P}_{k+1}^+)}{\partial\mat{K}} &=
    -2\mat{P}_{k+1}^{-\mathsf{T}}\mat{C}_{k+1}\T + 2\mat{K}_{k+1}\mat{S}_{k+1}
    \\
  \frac{\partial\tr(\mat{P}_{k+1}^+)}{\partial\mat{K}} &=
    -2\mat{P}_{k+1}^-\mat{C}_{k+1}\T + 2\mat{K}_{k+1}\mat{S}_{k+1} \\
  \mat{0} &= -2\mat{P}_{k+1}^-\mat{C}_{k+1}\T + 2\mat{K}_{k+1}\mat{S}_{k+1} \\
  2\mat{K}_{k+1}\mat{S}_{k+1} &= 2\mat{P}_{k+1}^-\mat{C}_{k+1}\T \\
  \mat{K}_{k+1}\mat{S}_{k+1} &= \mat{P}_{k+1}^-\mat{C}_{k+1}\T \\
  \mat{K}_{k+1} &= \mat{P}_{k+1}^-\mat{C}_{k+1}\T\mat{S}_{k+1}^{-1}
\end{align*}

This is the optimal Kalman gain.

\subsubsection{Simplified \textit{a posteriori} estimate covariance update
  equation}

If the optimal Kalman gain is used, the \textit{a posteriori} estimate
covariance matrix update equation can be simplified. First, we'll manipulate the
equation for the optimal Kalman gain.
\begin{align*}
  \mat{K}_{k+1} &= \mat{P}_{k+1}^-\mat{C}_{k+1}\T\mat{S}_{k+1}^{-1} \\
  \mat{K}_{k+1}\mat{S}_{k+1} &= \mat{P}_{k+1}^-\mat{C}_{k+1}\T \\
  \mat{K}_{k+1}\mat{S}_{k+1}\mat{K}_{k+1}\T &=
    \mat{P}_{k+1}^-\mat{C}_{k+1}\T\mat{K}_{k+1}\T
\end{align*}

Now we'll substitute it into equation \eqref{eq:post_p_update}.
\begin{align*}
  \mat{P}_{k+1}^+ &=
    \mat{P}_{k+1}^- - \mat{P}_{k+1}^-\mat{C}_{k+1}\T\mat{K}_{k+1}\T -
    \mat{K}_{k+1}\mat{C}_{k+1}\mat{P}_{k+1}^- +
    \mat{K}_{k+1}\mat{S}_{k+1}\mat{K}_{k+1}\T \\
  \mat{P}_{k+1}^+ &=
    \mat{P}_{k+1}^- - \mat{P}_{k+1}^-\mat{C}_{k+1}\T\mat{K}_{k+1}\T -
    \mat{K}_{k+1}\mat{C}_{k+1}\mat{P}_{k+1}^- +
    (\mat{P}_{k+1}^-\mat{C}_{k+1}\T\mat{K}_{k+1}\T) \\
  \mat{P}_{k+1}^+ &=
    \mat{P}_{k+1}^- - \mat{K}_{k+1}\mat{C}_{k+1}\mat{P}_{k+1}^-
  \intertext{Factor out $\mat{P}_{k+1}^-$ to the right.}
  \mat{P}_{k+1}^+ &= (\mat{I} - \mat{K}_{k+1}\mat{C}_{k+1})\mat{P}_{k+1}^-
\end{align*}

\subsection{Predict and update equations}

Now that we've derived all the pieces we need, we can finally write all the
equations for a Kalman filter. Theorem \ref{thm:kalman_filter} shows the predict
and update steps for a Kalman filter at the $k^{th}$ timestep.

Intuitively, the predict step projects the error covariance forward in an
increasing parabolic shape because you become less certain in your state
estimate as you go longer since a measurement. In your state-space (think like
3D space but each state is an axis), the error ellipsoid grows over time.

The correct step decreases the error covariance again by injecting new
information. The input has no effect on this because it has no noise associated
with it. In fact, input cancels out in the Kalman filter expectation and
covariance update equation derivations.

A Kalman filter chooses the Kalman gain $\mat{K}_{k+1}$ in the correct equation
$\hat{\mat{x}}_{k+1}^+ = \hat{\mat{x}}_{k+1}^- + \mat{K}_{k+1}(\mat{y}_{k+1} -
\hat{\mat{y}}_{k+1})$ such that the eigenvalues of $\mat{P}$ are minimized. This
minimizes the error variances and thus the dimensions of the uncertainty
ellipsoid over time.

If the update period is constant, the predict and correct steps are run in a
loop as opposed to sporadically skipping correct steps, and the
$(\mat{A}, \mat{C})$ pair is detectable, the error covariance matrix $\mat{P}$
willl approach a steady-state. This steady-state $\mat{P}$ results in a
steady-state Kalman gain that can be used instead of the error covariance update
equations to conserve computational resources.

\index{state-space observers!Kalman filter!equations}
\begin{theorem}[Kalman filter]
  \label{thm:kalman_filter}
  \begin{align}
    \text{Predict step} \nonumber \\
    \hat{\mat{x}}_{k+1}^- &= \mat{A}\hat{\mat{x}}_k^+ + \mat{B} \mat{u}_k \\
    \mat{P}_{k+1}^- &= \mat{A} \mat{P}_k^- \mat{A}\T + \mat{Q} \\
    \text{Update step} \nonumber \\
    \mat{K}_{k+1} &=
      \mat{P}_{k+1}^- \mat{C}\T (\mat{C}\mat{P}_{k+1}^- \mat{C}\T +
      \mat{R})^{-1} \\
    \hat{\mat{x}}_{k+1}^+ &=
      \hat{\mat{x}}_{k+1}^- + \mat{K}_{k+1}(\mat{y}_{k+1} -
      \mat{C} \hat{\mat{x}}_{k+1}^- - \mat{D}\mat{u}_{k+1}) \\
    \mat{P}_{k+1}^+ &= (\mat{I} - \mat{K}_{k+1}\mat{C})\mat{P}_{k+1}^-
  \end{align}
  \begin{figurekey}
    \begin{tabular}{llll}
      $\mat{A}$ & system matrix & $\hat{\mat{x}}$ & state estimate vector \\
      $\mat{B}$ & input matrix       & $\mat{u}$ & input vector \\
      $\mat{C}$ & output matrix      & $\mat{y}$ & output vector \\
      $\mat{D}$ & feedthrough matrix & $\mat{Q}$ & process noise covariance \\
      $\mat{P}$ & error covariance matrix & $\mat{R}$ & measurement noise covariance \\
      $\mat{K}$ & Kalman gain matrix & &
    \end{tabular}
  \end{figurekey}

  where a superscript of minus denotes \textit{a priori} and plus denotes
  \textit{a posteriori} estimate (before and after update respectively).
\end{theorem}

$\mat{C}$, $\mat{D}$, $\mat{Q}$, and $\mat{R}$ from the equations derived
earlier are made constants here.
\begin{remark}
  To implement a discrete time Kalman filter from a continuous model, the model
  and continuous time $\mat{Q}$ and $\mat{R}$ matrices can be
  \glslink{discretization}{discretized} using theorem
  \ref{thm:linear_system_zoh}.
\end{remark}
\begin{booktable}
  \begin{tabular}{|ll|ll|}
    \hline
    \rowcolor{headingbg}
    \textbf{Matrix} & \textbf{Rows $\times$ Columns} &
    \textbf{Matrix} & \textbf{Rows $\times$ Columns} \\
    \hline
    $\mat{A}$ & states $\times$ states & $\hat{\mat{x}}$ & states $\times$ 1 \\
    $\mat{B}$ & states $\times$ inputs & $\mat{u}$ & inputs $\times$ 1 \\
    $\mat{C}$ & outputs $\times$ states & $\mat{y}$ & outputs $\times$ 1 \\
    $\mat{D}$ & outputs $\times$ inputs & $\mat{Q}$ & states $\times$ states \\
    $\mat{P}$ & states $\times$ states & $\mat{R}$ & outputs $\times$ outputs \\
    $\mat{K}$ & states $\times$ outputs & &
      \\
    \hline
  \end{tabular}
  \caption{Kalman filter matrix dimensions}
\end{booktable}

Unknown \glspl{state} in a Kalman filter are generally represented by a Wiener
(pronounced VEE-ner) process.\footnote{Explaining why we use the Wiener process
would require going much more in depth into stochastic processes and It\^{o}
calculus, which is outside the scope of this book.} This process has the
property that its variance increases linearly with time $t$.

\subsection{Setup}
\index{state-space observers!Kalman filter!setup}

\subsubsection{Equations to model}

The following example \gls{system} will be used to describe how to define and
initialize the matrices for a Kalman filter.

A robot is between two parallel walls. It starts driving from one wall to the
other at a velocity of $0.8$ cm/s and uses ultrasonic sensors to provide noisy
measurements of the distances to the walls in front of and behind it. To
estimate the distance between the walls, we will define three \glspl{state}:
robot position, robot velocity, and distance between the walls.
\begin{align}
  x_{k+1} &= x_k + v_k \Delta T \\
  v_{k+1} &= v_k \\
  x_{k+1}^w &= x_k^w
\end{align}

This can be converted to the following state-space \gls{model}.
\begin{equation}
  \mat{x}_k =
  \begin{bmatrix}
    x_k \\
    v_k \\
    x_k^w
  \end{bmatrix}
\end{equation}
\begin{equation}
  \mat{x}_{k+1} =
  \begin{bmatrix}
    1 & 1 & 0 \\
    0 & 0 & 0 \\
    0 & 0 & 1
  \end{bmatrix} \mat{x}_k +
  \begin{bmatrix}
    0 \\
    0.8 \\
    0
  \end{bmatrix} +
  \begin{bmatrix}
    0 \\
    0.1 \\
    0
  \end{bmatrix} w_k
\end{equation}

where the Gaussian random variable $w_k$ has a mean of $0$ and a variance of
$1$. The observation \gls{model} is
\begin{equation}
  \mat{y}_k =
  \begin{bmatrix}
    1 & 0 & 0 \\
    -1 & 0 & 1
  \end{bmatrix} \mat{x}_k + \theta_k
\end{equation}

where the covariance matrix of Gaussian measurement noise $\theta$ is a
$2 \times 2$ matrix with both diagonals $10$ cm$^2$.

The \gls{state} vector is usually initialized using the first measurement or
two. The covariance matrix entries are assigned by calculating the covariance of
the expressions used when assigning the state vector. Let $k = 2$.
\begin{align}
  \mat{Q} &= \begin{bmatrix}1\end{bmatrix} \\
  \mat{R} &=
  \begin{bmatrix}
    10 & 0 \\
    0 & 10
  \end{bmatrix} \\
  \hat{\mat{x}} &=
  \begin{bmatrix}
    \mat{y}_{k,1} \\
    (\mat{y}_{k,1} - \mat{y}_{k-1,1})/dt \\
    \mat{y}_{k,1} + \mat{y}_{k,2}
  \end{bmatrix} \\
  \mat{P} &=
  \begin{bmatrix}
    10 & 10/dt & 10 \\
    10/dt & 20/dt^2 & 10/dt \\
    10 & 10/dt & 20
  \end{bmatrix}
\end{align}

\subsubsection{Initial conditions}

To fill in the $\mat{P}$ matrix, we calculate the covariance of each combination
of \gls{state} variables. The resulting value is a measure of how much those
variables are correlated. Due to how the covariance calculation works out, the
covariance between two variables is the sum of the variance of matching terms
which aren't constants multiplied by any constants the two have. If no terms
match, the variables are uncorrelated and the covariance is zero.

In $\mat{P}_{11}$, the terms in $\mat{x}_1$ correlate with itself. Therefore,
$\mat{P}_{11}$ is $\mat{x}_1$'s variance, or $\mat{P}_{11} = 10$. For
$\mat{P}_{21}$, One term correlates between $\mat{x}_1$ and $\mat{x}_2$, so
$\mat{P}_{21} = \frac{10}{dt}$. The constants from each are simply multiplied
together. For $\mat{P}_{22}$, both measurements are correlated, so the variances
add together. Therefore, $\mat{P}_{22} = \frac{20}{dt^2}$. It continues in this
fashion until the matrix is filled up. Order doesn't matter for correlation, so
the matrix is symmetric.

\subsubsection{Selection of priors}

Choosing good priors is important for a well performing filter, even if little
information is known. This applies to both the measurement noise and the noise
\gls{model}. The act of giving a \gls{state} variable a large variance means you
know something about the \gls{system}. Namely, you aren't sure whether your
initial guess is close to the true \gls{state}. If you make a guess and specify
a small variance, you are telling the filter that you are very confident in your
guess. If that guess is incorrect, it will take the filter a long time to move
away from your guess to the true value.

\subsubsection{Covariance selection}

While one could assume no correlation between the \gls{state} variables and set
the covariance matrix entries to zero, this may not reflect reality. The Kalman
filter is still guarenteed to converge to the steady-state covariance after an
infinite time, but it will take longer than otherwise.

\subsubsection{Noise model selection}

We typically use a Gaussian distribution for the noise \gls{model} because the
sum of many independent random variables produces a normal distribution by the
central limit theorem. Kalman filters only require that the noise is zero-mean.
If the true value has an equal probability of being anywhere within a certain
range, use a uniform distribution instead. Each of these communicates
information regarding what you know about a system in addition to what you do
not.

\subsubsection{Process noise and measurement noise covariance selection}

Recall that the process noise covariance is $\mat{Q}$ and the measurement noise
covariance is $\mat{R}$. To tune the elements of these, it can be helpful to
take a collection of measurements, then run the Kalman filter on them offline to
evaluate its performance.

The diagonal elements of $\mat{R}$ are the variances of each measurement, which
can be easily determined from the offline measurements. The diagonal elements of
$\mat{Q}$ are the variances of each \gls{state}. They represent how much each
\gls{state} is expected to deviate from the \gls{model}.

Selecting $\mat{Q}$ is more difficult. If the data is trusted too much over the
model, one risks overfitting the data. One should balance estimating any hidden
\glspl{state} sufficiently with actually filtering out the noise.

\subsubsection{Modeling other noise colors}

The Kalman filter assumes a \gls{model} with zero-mean white noise. If the
\gls{model} is incomplete in some way, whether it's missing dynamics or assumes
an incorrect noise \gls{model}, the residual
$\widetilde{\mat{y}} = \mat{y} - \mat{C}\hat{\mat{x}}$ over time will have
probability characteristics not indicative of white noise (e.g., it isn't
zero-mean).

To handle other colors of noise in a Kalman filter, define that color of noise
in terms of white noise and augment the \gls{model} with it.

\subsection{Simulation}
\label{subsec:filter_simulation}

Figure \ref{fig:filter_all} shows the \gls{state} estimates and measurements of
the Kalman filter over time. Figure \ref{fig:filter_robot_pos} shows the
position estimate and variance over time. Figure \ref{fig:filter_wall_pos} shows
the wall position estimate and variance over time. Notice how the variances
decrease over time as the filter gathers more measurements. This means that the
filter becomes more confident in its \gls{state} estimates.

The final precisions in estimating the position of the robot and the wall are
the square roots of the corresponding elements in the covariance matrix. That
is, $0.5188$ m and $0.4491$ m respectively. They are smaller than the precision
of the raw measurements, $\sqrt{10} = 3.1623$ m. As expected, combining the
information from several measurements produces a better estimate than any one
measurement alone.
\begin{svg}{build/\chapterpath/kalman_filter_all}
  \caption{State estimates and measurements with Kalman filter}
  \label{fig:filter_all}
\end{svg}
\begin{svg}{build/\chapterpath/kalman_filter_robot_pos}
  \caption{Robot position estimate and variance with Kalman filter}
  \label{fig:filter_robot_pos}
\end{svg}
\begin{svg}{build/\chapterpath/kalman_filter_wall_pos}
  \caption{Wall position estimate and variance with Kalman filter}
  \label{fig:filter_wall_pos}
\end{svg}

\subsection{Kalman filter as Luenberger observer}
\index{state-space observers!Kalman filter!as Luenberger observer}

A Kalman filter can be represented as a Luenberger \gls{observer} by letting
$\mat{L} = \mat{A} \mat{K}_k$ (see appendix \ref{sec:deriv_kalman_luenberger}).
The Luenberger observer has a constant observer gain matrix $\mat{L}$, so the
steady-state Kalman gain is used to calculate it. We will demonstrate how to
find this shortly.

Kalman filter theory provides a way to place the poles of the Luenberger
observer optimally in the same way we placed the poles of the controller
optimally with LQR. The eigenvalues of the Kalman filter are
\begin{equation}
  \eig(\mat{A}(\mat{I} - \mat{K}_k\mat{C}))
\end{equation}

\subsubsection{Steady-state Kalman gain}

One may have noticed that the error covariance matrix can be updated
independently of the rest of the \gls{model}. The error covariance matrix tends
toward a steady-state value, and this matrix can be obtained via the discrete
algebraic Riccati equation. This can then be used to compute a steady-state
Kalman gain.

General matrix inverses like $\mat{S}^{-1}$ are expensive. We want to put
$\mat{K} = \mat{P}\mat{C}\T \mat{S}^{-1}$ into $\mat{A}\mat{x} = \mat{b}$ form
so we can solve it more efficiently.
\begin{align*}
  \mat{K} &= \mat{P}\mat{C}\T \mat{S}^{-1} \\
  \mat{K}\mat{S} &= \mat{P}\mat{C}\T \\
  (\mat{K}\mat{S})\T &= (\mat{P}\mat{C}\T)\T \\
  \mat{S}\T \mat{K}\T &= \mat{C}\mat{P}\T
  \intertext{The solution of $\mat{A}\mat{x} = \mat{b}$ can be found via
    $\mat{x} = \solve(\mat{A}, \mat{b})$.}
  \mat{K}\T &= \solve(\mat{S}\T, \mat{C}\mat{P}\T) \\
  \mat{K} &= \solve(\mat{S}\T, \mat{C}\mat{P}\T)\T
\end{align*}

Snippet \ref{lst:kalman} computes the steady-state Kalman gain matrix.
\begin{coderemote}{Python}{snippets/kalmd.py}
  \caption{Steady-state Kalman gain matrix calculation in Python}
  \label{lst:kalman}
\end{coderemote}

\input{\chapterpath/kalman-smoother}
\section{Extended Kalman filter}
\label{sec:ekf}
\index{nonlinear control!Extended Kalman filter}
\index{state-space observers!Extended Kalman filter}

In this book, we have covered the Kalman filter, which is the optimal unbiased
estimator for linear \glspl{system}. One method for extending it to nonlinear
systems is the extended Kalman filter.

The extended Kalman filter (EKF) \glslink{linearization}{linearizes} the
dynamics and measurement models during the prediction and correction steps
respectively. Then, the linear Kalman filter equations are used to compute the
error covariance matrix $\mat{P}$ and Kalman gain matrix $\mat{K}$.

Theorem \ref{thm:extended_kalman_filter} shows the predict and update steps for
an extended Kalman filter at the $k^{th}$ timestep.
\begin{theorem}[Extended Kalman filter]
  \label{thm:extended_kalman_filter}
  \begin{align}
    \text{Predict step} \nonumber \\
    \mat{A} &= \left.\frac{\partial f(\mat{x}, \mat{u})}{\partial \mat{x}}
               \right|_{\hat{\mat{x}}_k^+, \mat{u}_k}
      \quad\text{(linearize $f(\mat{x}, \mat{u})$)} \\
    \mat{A}_k &= e^{\mat{A}T} \qquad\text{(discretize $\mat{A}$)} \\
    \hat{\mat{x}}_{k+1}^- &= f(\hat{\mat{x}}_k^+, \mat{u}_k) \\
    \mat{P}_{k+1}^- &= \mat{A}_k \mat{P}_k^- \mat{A}_k\T + \mat{Q}_k \\
    \text{Update step} \nonumber \\
    \mat{C}_{k+1} &= \left.\frac{\partial h(\mat{x}, \mat{u})}{\partial \mat{x}}
                     \right|_{\hat{\mat{x}}_{k+1}^-, \mat{u}_{k+1}}
      \quad\text{(linearize $h(\mat{x}, \mat{u})$)} \\
    \mat{K}_{k+1} &= \mat{P}_{k+1}^- \mat{C}_{k+1}\T
      (\mat{C}_{k+1}\mat{P}_{k+1}^- \mat{C}_{k+1}\T + \mat{R}_{k+1})^{-1} \\
    \hat{\mat{x}}_{k+1}^+ &=
      \hat{\mat{x}}_{k+1}^- + \mat{K}_{k+1}(\mat{y}_{k+1} -
      h(\hat{\mat{x}}_{k+1}^-, \mat{u}_{k+1})) \\
    \mat{P}_{k+1}^+ &= (\mat{I} - \mat{K}_{k+1}\mat{C}_{k+1})\mat{P}_{k+1}^-
  \end{align}
  \begin{figurekey}
    \begin{tabular}{llll}
      $f(\mat{x}, \mat{u})$ & continuous dynamics model & $\hat{\mat{x}}$ & state estimate vector \\
      $h(\mat{x}, \mat{u})$ & measurement model & $\mat{u}$ & input vector \\
      $T$ & sample timestep duration & $\mat{y}$ & output vector \\
      $\mat{P}$ & error covariance matrix & $\mat{Q}$ & process noise covariance \\
      $\mat{K}$ & Kalman gain matrix & $\mat{R}$ & measurement noise covariance \\
    \end{tabular}
  \end{figurekey}

  where a superscript of minus denotes \textit{a priori} and plus denotes
  \textit{a posteriori} estimate (before and after update respectively).
\end{theorem}
\begin{remark}
  To implement a discrete time Kalman filter from a continuous model, the model
  and continuous time $\mat{Q}$ and $\mat{R}$ matrices can be
  \glslink{discretization}{discretized} using theorem
  \ref{thm:linear_system_zoh}.
\end{remark}
\begin{booktable}
  \begin{tabular}{|ll|ll|}
    \hline
    \rowcolor{headingbg}
    \textbf{Matrix} & \textbf{Rows $\times$ Columns} &
    \textbf{Matrix} & \textbf{Rows $\times$ Columns} \\
    \hline
    $\mat{A}$ & states $\times$ states & $\hat{\mat{x}}$ & states $\times$ 1 \\
    $\mat{C}$ & outputs $\times$ states & $\mat{u}$ & inputs $\times$ 1 \\
    $\mat{P}$ & states $\times$ states & $\mat{y}$ & outputs $\times$ 1 \\
    $\mat{K}$ & states $\times$ outputs & $\mat{Q}$ & states $\times$ states \\
              &                         & $\mat{R}$ & outputs $\times$ outputs \\
    \hline
  \end{tabular}
  \caption{Extended Kalman filter matrix dimensions}
\end{booktable}

\section{Unscented Kalman filter}
\label{sec:ukf}
\index{nonlinear control!Unscented Kalman filter}
\index{state-space observers!Unscented Kalman filter}

In this book, we have covered the Kalman filter, which is the optimal unbiased
estimator for linear \glspl{system}. One method for extending it to nonlinear
systems is the unscented Kalman filter.

The unscented Kalman filter (UKF) propagates carefully chosen points called
sigma points through the nonlinear state and measurement models to obtain
estimates of the true covariances (as opposed to linearized versions of them).
We recommend reading Roger Labbe's book \textit{Kalman and Bayesian Filters in
Python} for more on
UKFs.\footnote{\url{https://github.com/rlabbe/Kalman-and-Bayesian-Filters-in-Python/blob/master/10-Unscented-Kalman-Filter.ipynb}}

The original paper on the UKF is also an option \cite{bib:ukf}. See also the
equations for van der Merwe's sigma point algorithm \cite{bib:ukf_sigma_points}.
Here's a paper on a quaternion-based Unscented Kalman filter for orientation
tracking \cite{bib:ukf_state_tracking}.

Here's an interview about the origin of the UKF with its creator
\cite{bib:first-hand_the_ut}.

\subsection{Square-root UKF}

The UKF uses a matrix square root (Cholesky decomposition) to compute the
scaling for the sigma points. This operation can introduce numerical
instability. The square-root UKF avoids this by propagating the square-root of
the error covariance directly instead of the error covariance; the Cholesky
decomposition is replaced with a QR decomposition and a Cholesky rank-1 update.

See the original paper \cite{bib:ukf_square_root} and this implementation
tutorial \cite{bib:ukf_square_root_tutorial}.

\input{\chapterpath/mmae}
