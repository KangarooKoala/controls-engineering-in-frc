\section{Derivative term}

The \textit{Derivative} term drives the velocity error to zero.
\begin{definition}[PD controller]
  \begin{equation}
    u(t) = K_p e(t) + K_d \frac{de}{dt}
  \end{equation}

  where $K_p$ is the proportional gain, $K_d$ is the derivative gain, and $e(t)$
  is the error at the current time $t$.
\end{definition}

Figure \ref{fig:pd_ctrl_diag} shows a block diagram for a \gls{system}
controlled by a PD controller.
\begin{bookfigure}
  \begin{tikzpicture}[auto, >=latex']
    \fontsize{9pt}{10pt}

    % Place the blocks
    \node [name=input] {$r(t)$};
    \node [sum, right=0.5cm of input] (errorsum) {};
    \node [coordinate, right=0.75cm of errorsum] (branch) {};
    \node [coordinate, right=1.0cm of branch] (I) {};
    \node [block, above=0.25cm of I] (P) { $K_p e(t)$ };
    \node [block, below=0.25cm of I] (D) { $K_d \frac{de(t)}{dt}$ };
    \node [sum, right=1.0cm of I] (ctrlsum) {};
    \node [block, right=0.75cm of ctrlsum] (plant) {Plant};
    \node [right=0.75cm of plant] (output) {};
    \node [coordinate, below=0.5cm of D] (measurements) {};

    % Connect the nodes
    \draw [arrow] (input) -- node[pos=0.9] {$+$} (errorsum);
    \draw [-] (errorsum) -- node {$e(t)$} (branch);
    \draw [arrow] (branch) |- (P);
    \draw [arrow] (branch) |- (D);
    \draw [arrow] (P) -| node[pos=0.95, left] {$+$} (ctrlsum);
    \draw [arrow] (D) -| node[pos=0.95, right] {$+$} (ctrlsum);
    \draw [arrow] (ctrlsum) -- node {$u(t)$} (plant);
    \draw [arrow] (plant) -- node [name=y] {$y(t)$} (output);
    \draw [-] (y) |- (measurements);
    \draw [arrow] (measurements) -| node[pos=0.99, right] {$-$} (errorsum);
  \end{tikzpicture}

  \caption{PD controller block diagram}
  \label{fig:pd_ctrl_diag}
\end{bookfigure}
\begin{svg}{build/\chapterpath/pd_controller}
  \caption{PD controller on an elevator}
  \label{fig:pd_controller}
\end{svg}

A PD controller has a proportional controller for position ($K_p$) and a
proportional controller for velocity ($K_d$). The velocity \gls{setpoint} is
implicitly provided by how the position \gls{setpoint} changes over time. Figure
\ref{fig:pd_controller} shows an example for an elevator.

To prove a PD controller is just two proportional controllers, we will rearrange
the equation for a PD controller.
\begin{align*}
  u_k &= K_p e_k + K_d \frac{e_k - e_{k-1}}{\Delta t}
  \intertext{where $u_k$ is the \gls{control input} at timestep $k$, $e_k$ is
    the \gls{error} at timestep $k$, and $\Delta t$ is the timestep duration.
    $e_k$ is defined as $e_k = r_k - y_k$ where $r_k$ is the \gls{setpoint} at
    timestep $k$ and $y_k$ is the \gls{output} at timestep $k$.}
  u_k &= K_p (r_k - y_k) +
    K_d \frac{(r_k - y_k) - (r_{k-1} - y_{k-1})}{\Delta t} \\
  u_k &= K_p (r_k - y_k) + K_d \frac{r_k - y_k - r_{k-1} + y_{k-1}}{\Delta t} \\
  u_k &= K_p (r_k - y_k) + K_d \frac{r_k - r_{k-1} - y_k + y_{k-1}}{\Delta t} \\
  u_k &= K_p (r_k - y_k) +
    K_d \frac{(r_k - r_{k-1}) - (y_k - y_{k-1})}{\Delta t} \\
  u_k &= K_p (r_k - y_k) + K_d \left(\frac{r_k - r_{k-1}}{\Delta t} -
    \frac{y_k - y_{k-1}}{\Delta t}\right)
\end{align*}

Notice how $\frac{r_k - r_{k-1}}{\Delta t}$ is the velocity of the
\gls{setpoint} and $\frac{y_k - y_{k-1}}{\Delta t}$ is the estimated velocity of
the \gls{system}. This means the $K_d$ term of the PD controller drives the
estimated velocity to the \gls{setpoint} velocity.

If the \gls{setpoint} is constant, the implicit velocity \gls{setpoint} is zero,
so the $K_d$ term slows the \gls{system} down if it's moving. This acts like a
``software-defined damper". These are commonly seen on door closers, and their
damping force increases linearly with velocity.
