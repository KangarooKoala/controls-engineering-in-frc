\section{Linear time-varying unicycle controller}

One can also create a linear time-varying controller with a cascaded control
architecture, where the outer layer consumes a pose command and produces
unicycle velocity commands, and the inner layer consumes unicycle velocity
commands and produces wheel motor voltages.

The change in global pose for a unicycle is defined by the following three
equations.
\begin{align*}
  \dot{x} &= v\cos\theta \\
  \dot{y} &= v\sin\theta \\
  \dot{\theta} &= \omega
\end{align*}

Here's the model as a vector function where
$\mat{x} = \begin{bmatrix} x & y & \theta \end{bmatrix}\T$ and
$\mat{u} = \begin{bmatrix} v & \omega \end{bmatrix}\T$.
\begin{equation}
  f(\mat{x}, \mat{u}) =
  \begin{bmatrix}
    v\cos\theta \\
    v\sin\theta \\
    \omega
  \end{bmatrix}
\end{equation}

To create an LQR, we need to linearize this.
\begin{equation*}
  \frac{\partial f(\mat{x}, \mat{u})}{\partial\mat{x}} =
  \begin{bmatrix}
    0 & 0 & -v\sin\theta \\
    0 & 0 & v\cos\theta \\
    0 & 0 & 0
  \end{bmatrix}
  \quad
  \frac{\partial f(\mat{x}, \mat{u})}{\partial\mat{u}} =
  \begin{bmatrix}
    \cos\theta & 0 \\
    \sin\theta & 0 \\
    0 & 1
  \end{bmatrix}
\end{equation*}

We're going to make a cross-track error controller, so we'll apply a clockwise
rotation matrix to the global tracking error to transform it into the robot's
coordinate frame. Since the cross-track error is always measured from the
robot's coordinate frame, the \gls{model} used to compute the LQR should be
linearized around $\theta = 0$ at all times.
\begin{equation*}
  \begin{array}{ll}
    \mat{A} =
    \begin{bmatrix}
      0 & 0 & -v\sin 0 \\
      0 & 0 & v\cos 0 \\
      0 & 0 & 0
    \end{bmatrix} &
    \mat{B} =
    \begin{bmatrix}
      \cos 0 & 0 \\
      \sin 0 & 0 \\
      0 & 1
    \end{bmatrix} \\
    \mat{A} =
    \begin{bmatrix}
      0 & 0 & 0 \\
      0 & 0 & v \\
      0 & 0 & 0
    \end{bmatrix} &
    \mat{B} =
    \begin{bmatrix}
      1 & 0 \\
      0 & 0 \\
      0 & 1
    \end{bmatrix}
  \end{array}
\end{equation*}

Therefore,
\begin{theorem}[Linear time-varying unicycle controller]
  \label{thm:linear_time-varying_unicycle_controller}

  Let the unicycle dynamics be
  $\dot{\mat{x}} = f(\mat{x}, \mat{u}) =
  \begin{bmatrix}
    v\cos\theta &
    v\sin\theta &
    \omega
  \end{bmatrix}\T$ where
  \begin{equation*}
    \mat{x} =
    \begin{bmatrix}
      x \\
      y \\
      \theta
    \end{bmatrix} =
    \begin{bmatrix}
      \text{x position} \\
      \text{y position} \\
      \text{heading}
    \end{bmatrix}
    \quad
    \mat{u} =
    \begin{bmatrix}
      v \\
      \omega
    \end{bmatrix} =
    \begin{bmatrix}
      \text{linear velocity} \\
      \text{angular velocity}
    \end{bmatrix}
  \end{equation*}
  \begin{align*}
    \mat{A} &= \left.
      \frac{\partial f(\mat{x}, \mat{u})}{\partial\mat{x}}
    \right|_{\theta = 0} =
    \begin{bmatrix}
      0 & 0 & 0 \\
      0 & 0 & v \\
      0 & 0 & 0
    \end{bmatrix} \\
    \mat{B} &= \left.
      \frac{\partial f(\mat{x}, \mat{u})}{\partial\mat{u}}
    \right|_{\theta = 0} =
    \begin{bmatrix}
      1 & 0 \\
      0 & 0 \\
      0 & 1
    \end{bmatrix}
  \end{align*}
  The linear time-varying unicycle controller is
  \begin{equation}
    \mat{u} = \mat{K}
    \begin{bmatrix}
      \cos\theta & \sin\theta & 0 \\
      -\sin\theta & \cos\theta & 0 \\
      0 & 0 & 1
    \end{bmatrix}
    (\mat{r} - \mat{x})
  \end{equation}

  At each timestep, the LQR controller gain $\mat{K}$ is computed for the
  $(\mat{A}, \mat{B})$ pair evaluated at the current input.
\end{theorem}

With the \gls{model} in theorem
\ref{thm:linear_time-varying_unicycle_controller}, $y$ is uncontrollable at
$v = 0$ because nonholonomic drivetrains are unable to move sideways. Some DARE
solvers throw errors in this case, but one can avoid it by linearizing the model
around a slightly nonzero velocity instead.

The controller in theorem \ref{thm:linear_time-varying_unicycle_controller}
results in figures \ref{fig:ltv_unicycle_traj_xy} and
\ref{fig:ltv_unicycle_traj_response}.
\begin{bookfigure}
  \begin{minisvg}{2}{build/\chapterpath/ltv_unicycle_traj_xy}
    \caption{Linear time-varying unicycle controller x-y plot}
    \label{fig:ltv_unicycle_traj_xy}
  \end{minisvg}
  \hfill
  \begin{minisvg}{2}{build/\chapterpath/ltv_unicycle_traj_response}
    \caption{Linear time-varying unicycle controller response}
    \label{fig:ltv_unicycle_traj_response}
  \end{minisvg}
\end{bookfigure}
