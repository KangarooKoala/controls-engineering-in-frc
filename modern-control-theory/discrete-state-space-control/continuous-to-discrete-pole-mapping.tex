\section{Continuous to discrete pole mapping}

When a continuous system is discretized, its poles in the LHP map to the inside
of a unit circle. Table \ref{tab:c2d_mapping} contains a few common points and
figure \ref{fig:c2d_mapping} shows the mapping visually.
\begin{booktable}
  \begin{tabular}{|cc|}
    \hline
    \rowcolor{headingbg}
    \textbf{Continuous} & \textbf{Discrete} \\
    \hline
    $(0, 0)$ & $(1, 0)$ \\
    imaginary axis & edge of unit circle \\
    $(-\infty, 0)$ & $(0, 0)$ \\
    \hline
  \end{tabular}
  \caption{Mapping from continuous to discrete}
  \label{tab:c2d_mapping}
\end{booktable}
\begin{bookfigure}
  \begin{minisvg}{2}{build/figs/s_plane}
  \end{minisvg}
  \hfill
  \begin{minisvg}{2}{build/figs/z_plane}
  \end{minisvg}
  \caption{Mapping of complex plane from continuous (left) to discrete (right)}
  \label{fig:c2d_mapping}
\end{bookfigure}

\subsection{Discrete system stability}

Eigenvalues of a \gls{system} that are within the unit circle are stable. To
demonstrate this, consider the discrete system $x_{k + 1} = ax_k$ where $a$ is a
complex number. $|a| < 1$ will make $x_{k + 1}$ converge to zero.

\subsection{Discrete system behavior}

Figure \ref{fig:disc_impulse_response_eig} shows the \glspl{impulse response} in
the time domain for \glspl{system} with various pole locations in the complex
plane (real numbers on the x-axis and imaginary numbers on the y-axis). Each
response has an initial condition of $1$.
\begin{bookfigure}
  \begin{tikzpicture}[auto, >=latex']
  % \draw [help lines] (-4,-4) grid (4,4);

  % Draw main axes
  \draw[->] (-4,0) -- (4,0) node[below] {\small Re};
  \draw[->] (0,-4) -- (0,4) node[right] {\small Im};

  % Unit circle
  \draw[black] (0,0) circle (3cm);

  % Exponent for the given x plot coordinate:
  %
  %   exp(at) = x
  %   at = ln(x)
  %   a = ln(x)/t
  %
  % t = 50 ms

  % LHP integrator
  \drawdiscretecoordplot{-3cm - 0.166cm}{-0.75cm}{0.125cm}{0.44375cm}
    {(0,1) (0.05,-1) (0.1,1) (0.15,-1) (0.2,1) (0.25,-1) (0.3,1) (0.35,-1)
     (0.4,1) (0.45,-1) (0.5,1)}
  \drawpole{-3cm}{0cm}

  % LHP stable: xₖ₊₁ = −2/3xₖ
  \drawdiscretecoordplot{-2cm}{0.75cm}{0.125cm}{0.44375cm}
    {(0,1) (0.05,-0.667) (0.1,0.444) (0.15,-0.296) (0.2,0.198) (0.25,-0.132)
     (0.3,0.088) (0.35,-0.059) (0.4,0.039) (0.45,-0.026) (0.5,0.017)}
  \drawpole{-2cm}{0cm}

  % LHP stable: xₖ₊₁ = −1/3xₖ
  \drawdiscretecoordplot{-1cm + 0.166cm}{-0.75cm}{0.125cm}{0.44375cm}
    {(0,1) (0.05,-0.333) (0.1,0.111) (0.15,-0.037) (0.2,0.012) (0.25,-0.004)
     (0.3,0.001) (0.35,0) (0.4,0) (0.45,0) (0.5,0)}
  \drawpole{-1cm}{0cm}

  % LHP stable: xₖ₊₁ = (0.333 + 0.5i)xₖ
  \drawdiscretecoordplot{-1cm}{2.25cm}{0.125cm}{0.44375cm}
    {(0,1) (0.05,-0.333) (0.1,-0.139) (0.15,0.213) (0.2,-0.092) (0.25,-0.016)
     (0.3,0.044) (0.35,-0.023) (0.4,0) (0.45,0.009) (0.5,-0.006)}
  \drawpole{-1cm}{1.5cm}

  % LHP unstable: xₖ₊₁ = (−1 − 0.75i)xₖ
  \drawdiscretecoordplot{-2.25cm}{-2.25cm}{0.125cm}{0.44375cm}
    {(0,1) (0.05,-1) (0.1,0.4375) (0.15,0.6875) (0.2,-2.059) (0.25,3.043)
     (0.3,-2.869) (0.35,0.984) (0.4,2.515) (0.45,-6.568) (0.5,9.206)}
  \drawpole{-3cm}{-2.25cm}

  % RHP stable: exp(-10.192t) cos(19.665wt) (40/3*w for readability)
  %
  % a = ln(0.333 + 0.5i)/0.05 = -10.192 + 19.665i
  \drawdiscretetimeplot{1cm}{2.25cm}{0.125cm}{0.44375cm}{
    exp(-10.192 * \x) * cos(40/3 * 19.665 * deg(\x))}
  \drawpole{1cm}{1.5cm}

  % RHP stable: exp(-21.97t)
  %
  % a = ln(1/3)/0.05 = -21.97
  \drawdiscretetimeplot{1cm - 0.166cm}{-0.75cm}{0.125cm}{0.125cm}{exp(-21.97 * \x)}
  \drawpole{1cm}{0cm}

  % RHP stable: exp(-8.109t)
  %
  % a = ln(2/3)/0.05 = -8.109
  \drawdiscretetimeplot{2cm}{0.75cm}{0.125cm}{0.125cm}{exp(-8.109 * \x)}
  \drawpole{2cm}{0cm}

  % RHP marginally stable: cos(15.707wt)
  %
  % a = ln(√2 + √2i)/0.05 = 15.707i
  \drawdiscretetimeplot{3 * (0.707cm + 0.25cm)}{3 * 0.707cm + 0.375cm}{0.125cm}{0.44375cm}{cos(15.707 * deg(\x))}
  \drawpole{3 * 0.707cm}{3 * 0.707cm}

  % RHP integrator
  \drawdiscretetimeplot{3cm + 0.166cm}{-0.75cm}{0.125cm}{0.125cm}{exp(0 * \x)}
  \drawpole{3cm}{0cm}

  % RHP unstable: exp(4.463t) cos(-12.87wt)
  %
  % a = ln(1 - 0.75i)/0.05 = 4.463 - 12.87i
  \drawdiscretetimeplot{2.25cm}{-2.25cm}{0.125cm}{0.44375cm}{exp(4.463 * \x) * cos(-12.87 * deg(\x))}
  \drawpole{3cm}{-2.25cm}

  % LHP and RHP labels
  \draw (-3.5,1.5) node {LHP};
  \draw (3.5,1.5) node {RHP};

  % Stable and unstable labels
  \fill[white] (-0.5,-2.25) rectangle (0.5,-1.75);
  \draw (0,-2) node {\small Stable};
  \draw (2.5,3.5) node {\small Unstable};
  \draw (-2.5,3.5) node {\small Unstable};
  \draw (-2.5,-3.5) node {\small Unstable};
  \draw (2.5,-3.5) node {\small Unstable};
\end{tikzpicture}

  \caption{Discrete impulse response vs pole location}
  \label{fig:disc_impulse_response_eig}
\end{bookfigure}

As $\omega$ increases in $s = j\omega$, a pole in the discrete plane moves
around the perimeter of the unit circle. Once it hits $\frac{\omega_s}{2}$ (half
the sampling frequency) at $(-1, 0)$, the pole wraps around. This is due to
poles faster than the sample frequency folding down to below the sample
frequency (that is, higher frequency signals \textit{alias} to lower frequency
ones).

Placing the poles at $(0, 0)$ produces a \textit{deadbeat controller}. An
$\rm N^{th}$-order deadbeat controller decays to the \gls{reference} in N
timesteps. While this sounds great, there are other considerations like
\gls{control effort}, \gls{robustness}, and \gls{noise immunity}.

Poles in the left half-plane cause jagged outputs because the frequency of the
\gls{system} dynamics is above the Nyquist frequency (twice the sample
frequency). The \glslink{discretization}{discretized} signal doesn't have enough
samples to reconstruct the continuous \gls{system}'s dynamics. See figures
\ref{fig:continuous_oscillations_1p} and \ref{fig:discrete_oscillations_2p} for
examples.
\begin{bookfigure}
  \begin{minisvg}{2}{build/\chapterpath/z_oscillations_1p}
    \caption{Single poles in various locations in discrete plane}
    \label{fig:continuous_oscillations_1p}
  \end{minisvg}
  \hfill
  \begin{minisvg}{2}{build/\chapterpath/z_oscillations_2p}
    \caption{Complex conjugate poles in various locations in discrete plane}
    \label{fig:discrete_oscillations_2p}
  \end{minisvg}
\end{bookfigure}

\subsection{Nyquist frequency}
\index{digital signal processing!Nyquist frequency}
\index{digital signal processing!aliasing}

To completely reconstruct a signal, the Nyquist-Shannon sampling theorem states
that it must be sampled at a frequency at least twice the maximum frequency it
contains. The highest frequency a given sample rate can capture is called the
Nyquist frequency, which is half the sample frequency. This is why recorded
audio is sampled at $44.1$ kHz. The maximum frequency a typical human can hear
is about $20$ kHz, so the Nyquist frequency is $20$ kHz and the minimum sampling
frequency is $40$ kHz. ($44.1$ kHz in particular was chosen for unrelated
historical reasons.)

Frequencies above the Nyquist frequency are folded down across it. The higher
frequency and the folded down lower frequency are said to alias each
other.\footnote{The aliases of a frequency $f$ can be expressed as
$f_{alias}(N) \stackrel{def}{=} |f - Nf_s|$. For example, if a $200$ Hz sine
wave is sampled at $150$ Hz, the \gls{observer} will see a $50$ Hz signal
instead of a $200$ Hz one.} Figure \ref{fig:c2d_aliasing} demonstrates
aliasing.
\begin{svg}{build/\chapterpath/aliasing}
  \caption{The original signal is a $1.5$ Hz sine wave, which means its Nyquist
    frequency is $1.5$ Hz. The signal is being sampled at $2$ Hz, so the aliased
    signal is a $0.5$ Hz sine wave.}
    \label{fig:c2d_aliasing}
\end{svg}

The effect of these high-frequency aliases can be reduced with a low-pass filter
(called an anti-aliasing filter in this application).
