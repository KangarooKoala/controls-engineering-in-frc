\section{What is a dynamical system?}

A dynamical system is a \gls{system} whose motion varies according to a set of
differential equations. A dynamical system is considered \textit{linear} if the
differential equations describing its dynamics consist only of linear operators.
Linear operators are things like constant gain multiplications, derivatives, and
integrals. You can define reasonably accurate linear \glspl{model} for pretty
much everything you'll see in FRC with just those relations.

But let's say you have a DC motor hooked up to a power supply and you applied a
constant voltage to it from rest. The motor approaches a steady-state angular
velocity, but the shape of the angular velocity curve over time isn't a line. In
fact, it's a decaying exponential curve akin to
\begin{equation*}
  \omega = \omega_{max}\left(1 - e^{-t}\right)
\end{equation*}

where $\omega$ is the angular velocity and $\omega_{max}$ is the maximum angular
velocity. If DC motors are said to behave linearly, then why is this?

Linearity refers to a \gls{system}'s equations of motion, not its time domain
response. The equation defining the motor's change in angular velocity over time
looks like
\begin{equation*}
  \dot{\omega} = -a\omega + bV
\end{equation*}

where $\dot{\omega}$ is the derivative of $\omega$ with respect to time, $V$ is
the input voltage, and $a$ and $b$ are constants specific to the motor. This
equation, unlike the one shown before, is actually linear because it only
consists of multiplications and additions relating the \gls{input} $V$ and
current \gls{state} $\omega$.

Also of note is that the relation between the input voltage and the angular
velocity of the output shaft is a linear regression. You'll see why if you model
a DC motor as a voltage source and generator producing back-EMF (in the equation
above, $bV$ corresponds to the voltage source and $-a\omega$ corresponds to the
back-EMF). As you increase the input voltage, the back-EMF increases linearly
with the motor's angular velocity. If there was a friction term that varied with
the angular velocity squared (air resistance is one example), the relation from
input to output would be a curve. Friction that scales with just the angular
velocity would result in a lower maximum angular velocity, but because that term
can be lumped into the back-EMF term, the response is still linear.
