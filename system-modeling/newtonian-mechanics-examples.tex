\chapterimage{system-modeling.jpg}{Hills by northbound freeway between Santa Maria and Ventura}

\chapter{Newtonian mechanics examples}

A \gls{model} is a set of differential equations describing how the \gls{system}
behaves over time. There are two common approaches for developing them.
\begin{enumerate}
  \item Collecting data on the physical system's behavior and performing
    \gls{system} identification with it.
  \item Using physics to derive the \gls{system}'s model from first principles.
\end{enumerate}

This chapter covers the second approach using Newtonian mechanics.

The \glspl{model} derived here should cover most types of motion seen on an FRC
robot. Furthermore, they can be easily tweaked to describe many types of
mechanisms just by pattern-matching. There's only so many ways to hook up a mass
to a motor in FRC. The flywheel \gls{model} can be used for spinning mechanisms,
the elevator \gls{model} can be used for spinning mechanisms transformed to
linear motion, and the single-jointed arm \gls{model} can be used for rotating
servo mechanisms (it's just the flywheel \gls{model} augmented with a position
\gls{state}).

These \glspl{model} assume all motor controllers driving DC motors are set to
brake mode instead of coast mode. Brake mode behaves the same as coast mode
except where the applied voltage is zero. In brake mode, the motor leads are
shorted together to prevent movement. In coast mode, the motor leads are an open
circuit.

\renewcommand*{\chapterpath}{\partpath/newtonian-mechanics-examples}
\section{DC motor}
\label{sec:dc_motor}

We will be deriving a first-order \gls{model} for a DC motor. A second-order
\gls{model} would include the inductance of the motor windings as well, but
we're assuming the time constant of the inductor is small enough that its affect
on the \gls{model} behavior is negligible for FRC use cases (see subsection
\ref{subsec:do_flywheels_need_pd_control} for a demonstration of this for a real
DC motor).
\begin{remark}
  For the brushless motor commutation methods currently available in FRC
  (trapezoidal commutation, field-oriented control), brushed and brushless DC
  motors have the same dynamics. However, more advanced commutation methods can
  break the linear back-EMF assumption of the brushed motor model.
\end{remark}

The first-order \gls{model} will only require numbers from the motor's
datasheet. The second-order \gls{model} would require measuring the motor
inductance as well, which generally isn't in the datasheet. It can be difficult
to measure accurately without the right equipment.

\subsection{Equations of motion}

The circuit for a DC motor is shown in figure
\ref{fig:dc_motor_circuit}.
\begin{bookfigure}
  \begin{tikzpicture}[auto, >=latex', circuit ee IEC,
                      set resistor graphic=var resistor IEC graphic]
    \node [opencircuit] (start) at (0,0) {};
    \node [] (V+) at (-0.5,0) { $+$ };
    \node [opencircuit] (end) at (0,-3.5) {};
    \node [] (V-) at (-0.5,-3.5) { $-$ };
    \node [coordinate] (topright) at (2.5,0) {};
    \node [coordinate] (bottomright) at (2.5,-3.5) {};
    \node [] at (0, -1.75) { $V$ };
    \draw (start) to (topright)
                  to [resistor={near start, info'={ $R$ }},
                      voltage source={near end, direction info'={<-},
                      info={ $V_{emf}=\frac{\omega}{K_v}$ }}] (bottomright)
                  to (end);
  \end{tikzpicture}

  \caption{DC motor circuit}
  \label{fig:dc_motor_circuit}
\end{bookfigure}

$V$ is the voltage applied to the motor, $I$ is the current through the motor in
Amps, $R$ is the resistance across the motor in Ohms, $\omega$ is the angular
velocity of the motor in radians per second, and $K_v$ is the angular velocity
constant in radians per second per Volt. This circuit reflects the following
relation.
\begin{equation}
  V = IR + \frac{\omega}{K_v} \label{eq:motor_V}
\end{equation}

The mechanical relation for a DC motor is
\begin{equation}
  \tau = K_t I
\end{equation}

where $\tau$ is the torque produced by the motor in Newton-meters and $K_t$ is
the torque constant in Newton-meters per Amp. Therefore
\begin{equation*}
  I = \frac{\tau}{K_t}
\end{equation*}

Substitute this into equation \eqref{eq:motor_V}.

\index{FRC models!DC motor equations}
\begin{equation}
  V = \frac{\tau}{K_t} R + \frac{\omega}{K_v} \label{eq:motor_tau_V}
\end{equation}

\subsection{Calculating constants}

A typical motor's datasheet should include graphs of the motor's measured torque
and current for different angular velocities for a given voltage applied to the
motor. Figure \ref{fig:motor_data} is an example. An FRC motor's datasheet can
be found on its vendor's website.
\begin{svg}{build/\chapterpath/motor_data}
  \caption{Example motor datasheet for 775pro}
  \label{fig:motor_data}
\end{svg}

\subsubsection{Torque constant $K_t$}
\begin{align}
  \tau &= K_t I \nonumber \\
  K_t &= \frac{\tau}{I} \nonumber \\
  K_t &= \frac{\tau_{stall}}{I_{stall}}
\end{align}

where $\tau_{stall}$ is the stall torque and $I_{stall}$ is the stall current of
the motor from its datasheet.

\subsubsection{Resistance $R$}

Recall equation \eqref{eq:motor_V}.
\begin{align}
  V &= IR + \frac{\omega}{K_v} \nonumber
  \intertext{When the motor is stalled, $\omega = 0$.}
  V &= I_{stall} R \nonumber \\
  R &= \frac{V}{I_{stall}}
\end{align}

where $I_{stall}$ is the stall current of the motor and $V$ is the voltage
applied to the motor at stall.

\subsubsection{Angular velocity constant $K_v$}

Recall equation \eqref{eq:motor_V}.
\begin{align}
  V &= IR + \frac{\omega}{K_v} \nonumber \\
  V - IR &= \frac{\omega}{K_v} \nonumber \\
  K_v &= \frac{\omega}{V - IR} \nonumber
  \intertext{When the motor is spinning under no load,}
  K_v &= \frac{\omega_{free}}{V - I_{free}R}
\end{align}

where $\omega_{free}$ is the angular velocity of the motor under no load (also
known as the free speed), and $V$ is the voltage applied to the motor when it's
spinning at $\omega_{free}$, and $I_{free}$ is the current drawn by the motor
under no load.
\begin{remark}
  To model a mechanism with several identical motors in one gearbox, multiply
  the stall torque, stall current, and free current by the number of motors $N$.
  $K_t$ and $K_v$ will be the same because $N$ cancels out, but $R$ will be
  divided by $N$. This multiplies the acceleration contribution of each model
  term by $N$.
\end{remark}

\subsection{Current limiting}

Current limiting of a DC motor reduces the maximum input voltage to avoid
exceeding a current threshold. Predictive current limiting uses a projected
estimate of the current, so the voltage is reduced before the current threshold
is exceeded. Reactive current limiting uses an actual current measurement, so
the voltage is reduced after the current threshold is exceeded.

The following pseudocode demonstrates each type of current limiting.
\begin{code}
  \begin{lstlisting}[style=customPython]
# Normal feedback control
V = K @ (r - x)

# Calculations for predictive current limiting
omega = angular_velocity_measurement
I = V / R - omega / (Kv * R)

# Calculations for reactive current limiting
I = current_measurement
omega = Kv * V - I * R * Kv  # or can be angular velocity measurement

# If predicted/actual current above max, limit current by reducing voltage
if I > I_max:
    V = I_max * R + omega / Kv
  \end{lstlisting}
  \caption{Limits current of DC motor to $I_{max}$}
\end{code}

\section{Elevator}

This elevator consists of a DC motor attached to a pulley that drives a mass up
or down.
\begin{bookfigure}
  \input{figs/elevator-system-diagram}
  \caption{Elevator system diagram}
  \label{fig:elevator}
\end{bookfigure}

Gear ratios are written as output over input, so $G$ is greater than one in
figure \ref{fig:elevator}.

\subsection{Equations of motion}

We want to derive an equation for the carriage acceleration $a$ (derivative of
$v$) given an input voltage $V$, which we can integrate to get carriage velocity
and position.

First, we'll find a torque to substitute into the equation for a DC motor. Based
on figure \ref{fig:elevator}
\begin{equation}
  \tau_m G = \tau_p \label{eq:elevator_tau_m_ratio}
\end{equation}

where $G$ is the gear ratio between the motor and the pulley and $\tau_p$ is the
torque produced by the pulley.
\begin{equation}
  rF_m = \tau_p \label{eq:elevator_torque_pulley}
\end{equation}

where $r$ is the radius of the pulley. Substitute equation
\eqref{eq:elevator_tau_m_ratio} into $\tau_m$ in the DC motor equation
\eqref{eq:motor_tau_V}.
\begin{align}
  V &= \frac{\frac{\tau_p}{G}}{K_t} R + \frac{\omega_m}{K_v} \nonumber \\
  V &= \frac{\tau_p}{GK_t} R + \frac{\omega_m}{K_v} \nonumber
  \intertext{Substitute in equation \eqref{eq:elevator_torque_pulley} for
    $\tau_p$.}
  V &= \frac{rF_m}{GK_t} R + \frac{\omega_m}{K_v} \label{eq:elevator_Vinter1}
\end{align}

The angular velocity of the motor armature $\omega_m$ is
\begin{equation}
  \omega_m = G \omega_p \label{eq:elevator_omega_m_ratio}
\end{equation}

where $\omega_p$ is the angular velocity of the pulley. The velocity of the mass
(the elevator carriage) is
\begin{equation*}
  v = r \omega_p
\end{equation*}
\begin{equation}
  \omega_p = \frac{v}{r} \label{eq:elevator_omega_p}
\end{equation}

Substitute equation \eqref{eq:elevator_omega_p} into equation
\eqref{eq:elevator_omega_m_ratio}.
\begin{equation}
  \omega_m = G \frac{v}{r} \label{eq:elevator_omega_m}
\end{equation}

Substitute equation \eqref{eq:elevator_omega_m} into equation
\eqref{eq:elevator_Vinter1} for $\omega_m$.
\begin{align}
  V &= \frac{rF_m}{GK_t} R + \frac{G \frac{v}{r}}{K_v} \nonumber \\
  V &= \frac{RrF_m}{GK_t} + \frac{G}{rK_v} v \nonumber
  \intertext{Solve for $F_m$.}
  \frac{RrF_m}{GK_t} &= V - \frac{G}{rK_v} v \nonumber \\
  F_m &= \left(V - \frac{G}{rK_v} v\right) \frac{GK_t}{Rr} \nonumber \\
  F_m &= \frac{GK_t}{Rr} V - \frac{G^2K_t}{Rr^2 K_v} v
\end{align}

We need to find the acceleration of the elevator carriage. Note that
\begin{equation}
  \sum F = ma
\end{equation}

where $\sum F$ is the sum of forces applied to the elevator carriage, $m$ is the
mass of the elevator carriage in kilograms, and $a$ is the acceleration of the
elevator carriage.
\begin{align}
  ma &= F_m \nonumber \\
  ma &= \left(\frac{GK_t}{Rr} V - \frac{G^2K_t}{Rr^2 K_v} v\right) \nonumber \\
  a &= \frac{GK_t}{Rrm} V - \frac{G^2K_t}{Rr^2 mK_v} v \nonumber \\
  a &= -\frac{G^2K_t}{Rr^2 mK_v} v + \frac{GK_t}{Rrm} V  \label{eq:elevator_accel}
\end{align}
\begin{remark}
  Gravity is not part of the modeled dynamics because it complicates the
  state-space \gls{model} and the controller will behave well enough without it.
\end{remark}

This model will be converted to state-space notation in section
\ref{sec:ss_model_elevator}.

\section{Flywheel}
\label{sec:ss_model_flywheel}

This flywheel consists of a DC motor attached to a spinning mass of
non-negligible moment of inertia.
\begin{bookfigure}
  \input{figs/flywheel-system-diagram}
  \caption{Flywheel system diagram}
\end{bookfigure}

\subsection{Continuous state-space model}
\index{FRC models!flywheel equations}

By equation \eqref{eq:dot_omega_flywheel}
\begin{align*}
  \dot{\omega} &= -\frac{G^2 K_t}{K_v RJ} \omega + \frac{G K_t}{RJ} V
  \intertext{Factor out $\omega$ and $V$ into column vectors.}
  \dot{\begin{bmatrix}
    \omega
  \end{bmatrix}} &=
  \begin{bmatrix}
    -\frac{G^2 K_t}{K_v RJ}
  \end{bmatrix}
  \begin{bmatrix}
    \omega
  \end{bmatrix} +
  \begin{bmatrix}
    \frac{GK_t}{RJ}
  \end{bmatrix}
  \begin{bmatrix}
    V
  \end{bmatrix}
\end{align*}
\begin{theorem}[Flywheel state-space model]
  \begin{align*}
    \dot{\mat{x}} &= \mat{A} \mat{x} + \mat{B} \mat{u} \\
    \mat{y} &= \mat{C} \mat{x} + \mat{D} \mat{u}
  \end{align*}
  \begin{equation*}
    \mat{x} = \omega = \text{angular velocity}
    \quad
    \mat{y} = \omega = \text{angular velocity}
    \quad
    \mat{u} = V = \text{voltage}
  \end{equation*}
  \begin{align}
    \mat{A} &= -\frac{G^2 K_t}{K_v RJ} \\
    \mat{B} &= \frac{G K_t}{RJ} \\
    \mat{C} &= 1 \\
    \mat{D} &= 0
  \end{align}
\end{theorem}

\subsection{Model augmentation}

As per subsection \ref{subsec:input_error_estimation}, we will now augment the
\gls{model} so a $u_{error}$ state is added to the \gls{control input}.

The \gls{plant} and \gls{observer} augmentations should be performed before the
\gls{model} is \glslink{discretization}{discretized}. After the \gls{controller}
gain is computed with the unaugmented discrete \gls{model}, the controller may
be augmented. Therefore, the \gls{plant} and \gls{observer} augmentations assume
a continuous \gls{model} and the \gls{controller} augmentation assumes a
discrete \gls{controller}.
\begin{equation*}
  \mat{x} =
  \begin{bmatrix}
    \omega \\
    u_{error}
  \end{bmatrix}
  \quad
  \mat{y} = \omega
  \quad
  \mat{u} = V
\end{equation*}
\begin{equation}
  \mat{A}_{aug} =
  \begin{bmatrix}
    \mat{A} & \mat{B} \\
    0 & 0
  \end{bmatrix}
  \quad
  \mat{B}_{aug} =
  \begin{bmatrix}
    \mat{B} \\
    0
  \end{bmatrix}
  \quad
  \mat{C}_{aug} = \begin{bmatrix}
    \mat{C} & 0
  \end{bmatrix}
  \quad
  \mat{D}_{aug} = \mat{D}
\end{equation}
\begin{equation}
  \mat{K}_{aug} = \begin{bmatrix}
    \mat{K} & 1
  \end{bmatrix}
  \quad
  \mat{r}_{aug} = \begin{bmatrix}
    \mat{r} \\
    0
  \end{bmatrix}
\end{equation}

This will compensate for unmodeled dynamics such as projectiles slowing down the
flywheel.

\subsection{Simulation}

Python Control will be used to \glslink{discretization}{discretize} the
\gls{model} and simulate it. One of the frccontrol
examples\footnote{\url{https://github.com/calcmogul/frccontrol/blob/main/examples/flywheel.py}}
creates and tests a controller for it. Figure \ref{fig:flywheel_response} shows
the closed-loop \gls{system} response.
\begin{svg}{build/\chapterpath/flywheel_response}
  \caption{Flywheel response}
  \label{fig:flywheel_response}
\end{svg}

Notice how the \gls{control effort} when the \gls{reference} is reached is
nonzero. This is a plant inversion feedforward compensating for the \gls{system}
dynamics attempting to slow the flywheel down when no voltage is applied.

\subsection{Implementation}

C++ and Java implementations of this flywheel controller are available
online.\footnote{\url{https://github.com/wpilibsuite/allwpilib/blob/main/wpilibcExamples/src/main/cpp/examples/StateSpaceFlywheel/cpp/Robot.cpp}}%
\footnote{\url{https://github.com/wpilibsuite/allwpilib/blob/main/wpilibjExamples/src/main/java/edu/wpi/first/wpilibj/examples/statespaceflywheel/Robot.java}}

\subsection{Flywheel model without encoder}

In the FIRST Robotics Competition, we can get the current drawn for specific
channels on the power distribution panel. We can theoretically use this to
estimate the angular velocity of a DC motor without an encoder. We'll start with
the flywheel model derived earlier as equation \eqref{eq:dot_omega_flywheel}.
\begin{align*}
  \dot{\omega} &= \frac{G K_t}{RJ} V - \frac{G^2 K_t}{K_v RJ} \omega \\
  \dot{\omega} &= -\frac{G^2 K_t}{K_v RJ} \omega + \frac{G K_t}{RJ} V
  \intertext{Next, we'll derive the current $I$ as an output.}
  V &= IR + \frac{\omega}{K_v} \\
  IR &= V - \frac{\omega}{K_v} \\
  I &= -\frac{1}{K_v R} \omega + \frac{1}{R} V
\end{align*}

Therefore,
\begin{theorem}[Flywheel state-space model without encoder]
  \begin{align*}
    \dot{\mat{x}} &= \mat{A} \mat{x} + \mat{B} \mat{u} \\
    \mat{y} &= \mat{C} \mat{x} + \mat{D} \mat{u}
  \end{align*}
  \begin{equation*}
    \mat{x} = \omega = \text{angular velocity}
    \quad
    \mat{y} = I = \text{current}
    \quad
    \mat{u} = V = \text{voltage}
  \end{equation*}
  \begin{align}
    \mat{A} &= -\frac{G^2 K_t}{K_v RJ} \\
    \mat{B} &= \frac{G K_t}{RJ} \\
    \mat{C} &= -\frac{1}{K_v R} \\
    \mat{D} &= \frac{1}{R}
  \end{align}
\end{theorem}

Notice that in this \gls{model}, the \gls{output} doesn't provide any direct
measurements of the \gls{state}. To estimate the full \gls{state} (also known as
full observability), we only need the \glspl{output} to collectively include
linear combinations of every \gls{state}\footnote{While the flywheel model's
outputs are a linear combination of both the states and inputs, \glspl{input}
don't provide new information about the \glspl{state}. Therefore, they don't
affect whether the system is observable.}. We'll revisit this in chapter
\ref{ch:stochastic_control_theory} with an example that uses range measurements
to estimate an object's orientation.

The effectiveness of this \gls{model}'s \gls{observer} is heavily dependent on
the quality of the current sensor used. If the sensor's noise isn't zero-mean,
the \gls{observer} won't converge to the true \gls{state}.

\subsection{Voltage compensation}

To improve controller \gls{tracking}, one may want to use the voltage
renormalized to the power rail voltage to compensate for voltage drop when
current spikes occur. This can be done as follows.
\begin{equation}
  V = V_{cmd} \frac{V_{nominal}}{V_{rail}}
\end{equation}

where $V$ is the \gls{controller}'s new input voltage, $V_{cmd}$ is the old
input voltage, $V_{nominal}$ is the rail voltage when effects like voltage drop
due to current draw are ignored, and $V_{rail}$ is the real rail voltage.

To drive the \gls{model} with a more accurate voltage that includes voltage
drop, the reciprocal can be used.
\begin{equation}
  V = V_{cmd} \frac{V_{rail}}{V_{nominal}}
\end{equation}

where $V$ is the \gls{model}'s new input voltage. Note that if both the
\gls{controller} compensation and \gls{model} compensation equations are
applied, the original voltage is obtained. The \gls{model} input only drops from
ideal if the compensated \gls{controller} voltage saturates.

\subsection{Do flywheels need PD control?}
\label{subsec:do_flywheels_need_pd_control}
\index{PID control!flywheel (modern control)}

PID controllers typically control voltage to a motor in FRC independent of the
equations of motion of that motor. For position PID control, large values of
$K_p$ can lead to overshoot and $K_d$ is commonly used to reduce overshoots.
Let's consider a flywheel controlled with a standard PID controller. Why
wouldn't $K_d$ provide damping for velocity overshoots in this case?

PID control is designed to control second-order and first-order \glspl{system}
well. It can be used to control a lot of things, but struggles when given higher
order \glspl{system}. It has three degrees of freedom. Two are used to place the
two poles of the \gls{system}, and the third is used to remove steady-state
error. With higher order \glspl{system} like a one input, seven \gls{state}
\gls{system}, there aren't enough degrees of freedom to place the \gls{system}'s
poles in desired locations. This will result in poor control.

The math for PID doesn't assume voltage, a motor, etc. It defines an output
based on derivatives and integrals of its input. We happen to use it for motors
because it actually works pretty well for it because motors are second-order
\glspl{system}.

The following math will be in continuous time, but the same ideas apply to
discrete time. This is all assuming a velocity controller.

Our simple motor model hooked up to a mass is
\begin{align}
  V &= IR + \frac{\omega}{K_v} \label{eq:steady-state_error_ss_flywheel_1} \\
  \tau &= I K_t \label{eq:steady-state_error_ss_flywheel_2} \\
  \tau &= J \frac{d\omega}{dt} \label{eq:steady-state_error_ss_flywheel_3}
\end{align}

For an explanation of where these equations come from, read section
\ref{sec:dc_motor}.

First, we'll solve for $\frac{d\omega}{dt}$ in terms of $V$.

Substitute equation \eqref{eq:steady-state_error_ss_flywheel_2} into equation
\eqref{eq:steady-state_error_ss_flywheel_1}.
\begin{align}
  V &= IR + \frac{\omega}{K_v} \nonumber \\
  V &= \left(\frac{\tau}{K_t}\right) R + \frac{\omega}{K_v} \nonumber
  \intertext{Substitute in equation
    \eqref{eq:steady-state_error_ss_flywheel_3}.}
  V &= \frac{\left(J \frac{d\omega}{dt}\right)}{K_t} R + \frac{\omega}{K_v}
    \nonumber \\
  \intertext{Solve for $\frac{d\omega}{dt}$.}
  V &= \frac{J \frac{d\omega}{dt}}{K_t} R + \frac{\omega}{K_v} \nonumber \\
  V - \frac{\omega}{K_v} &= \frac{J \frac{d\omega}{dt}}{K_t} R \nonumber \\
  \frac{d\omega}{dt} &= \frac{K_t}{JR} \left(V - \frac{\omega}{K_v}\right)
    \nonumber \\
  \underbrace{\frac{d\omega}{dt}}_{\dot{\mat{x}}} &=
    \underbrace{-\frac{K_t}{JRK_v}}_{\mat{A}} \underbrace{\omega}_{\mat{x}} +
    \underbrace{\frac{K_t}{JR}}_{\mat{B}} \underbrace{V}_{\mat{u}}
\end{align}

There's one stable open-loop pole at $-\frac{K_t}{JRK_v}$. Let's try a simple P
controller.
\begin{align*}
  \mat{u} &= \mat{K} (\mat{r} - \mat{x}) \\
  V &= K_p (\omega_{goal} - \omega)
\end{align*}

Closed-loop models have the form
$\dot{\mat{x}} = (\mat{A} - \mat{B}\mat{K})\mat{x} + \mat{B}\mat{K}\mat{r}$.
Therefore, the closed-loop poles are the eigenvalues of
$\mat{A} - \mat{B}\mat{K}$.
\begin{align*}
  \dot{\mat{x}} &= (\mat{A} - \mat{B}\mat{K})\mat{x} + \mat{B}\mat{K}\mat{r}
    \\
  \dot{\omega} &= \left(\left(-\frac{K_t}{JRK_v}\right) -
    \left(\frac{K_t}{JR}\right)(K_p)\right)\omega +
    \left(\frac{K_t}{JR}\right)(K_p)(\omega_{goal}) \\
  \dot{\omega} &= -\left(\frac{K_t}{JRK_v} + \frac{K_t K_p}{JR}\right)\omega +
    \frac{K_t K_p}{JR}\omega_{goal}
\end{align*}

This closed-loop flywheel model has one pole at
$-\left(\frac{K_t}{JRK_v} + \frac{K_t K_p}{JR}\right)$. It therefore only needs
one P controller to place that pole anywhere on the real axis. A derivative
term is unnecessary on an ideal flywheel. It may compensate for unmodeled
dynamics such as accelerating projectiles slowing the flywheel down, but that
effect may also increase recovery time; $K_d$ drives the acceleration to zero in
the undesired case of negative acceleration as well as well as the actually
desired case of positive acceleration.

This analysis assumes that the motor is well coupled to the mass and that the
time constant of the inductor is small enough that it doesn't factor into the
motor equations. The latter is a pretty good assumption for a CIM motor with the
following constants: $J = 3.2284 \times 10^{-6}$ $kg$-$m^2$,
$b = 3.5077 \times 10^{-6}$ $N$-$m$-$s$, $K_e = K_t = 0.0181 \,V/rad/s$,
$R = 0.0902 \,\Omega$, and $L = 230$ μH. Notice the slight wiggle in figure
\ref{fig:cs_ss_highfreq_unstable_step} compared to figure
\ref{fig:cs_ss_highfreq_stable_step}. If more mass is added to the motor
armature, the response timescales increase and the inductance matters even less.
\begin{bookfigure}
  \begin{minisvg}{2}{build/figs/highfreq_unstable_step}
    \caption{Step response of second-order DC motor plant augmented with
      position ($L = 230$ μH)}
    \label{fig:cs_ss_highfreq_unstable_step}
  \end{minisvg}
  \hfill
  \begin{minisvg}{2}{build/figs/highfreq_stable_step}
    \caption{Step response of first-order DC motor plant augmented with position
      ($L = 0$ μH)}
    \label{fig:cs_ss_highfreq_stable_step}
  \end{minisvg}
\end{bookfigure}

Subsection \ref{subsec:input_error_estimation} covers a superior compensation
method that avoids zeroes in the \gls{controller}, doesn't act against the
desired control action, and facilitates better \gls{tracking}.

\section{Single-jointed arm}

This single-jointed arm consists of a DC motor attached to a pulley that spins a
straight bar in pitch.
\begin{bookfigure}
  \input{figs/single-jointed-arm-system-diagram}
  \caption{Single-jointed arm system diagram}
  \label{fig:single_jointed_arm}
\end{bookfigure}

Gear ratios are written as output over input, so $G$ is greater than one in
figure \ref{fig:single_jointed_arm}.

\subsection{Equations of motion}

We want to derive an equation for the arm angular acceleration
$\dot{\omega}_{arm}$ given an input voltage $V$, which we can integrate to get
arm angular velocity and angle.

We will start with the equation derived earlier for a DC motor, equation
\eqref{eq:motor_tau_V}.
\begin{align}
  V &= \frac{\tau_m}{K_t} R + \frac{\omega_m}{K_v} \nonumber
  \intertext{Solve for the angular acceleration. First, we'll rearrange the
    terms because from inspection, $V$ is the \gls{model} \gls{input},
    $\omega_m$ is the \gls{state}, and $\tau_m$ contains the angular
    acceleration.}
  V &= \frac{R}{K_t} \tau_m + \frac{1}{K_v} \omega_m \nonumber
  \intertext{Solve for $\tau_m$.}
  V &= \frac{R}{K_t} \tau_m + \frac{1}{K_v} \omega_m \nonumber \\
  \frac{R}{K_t} \tau_m &= V - \frac{1}{K_v} \omega_m \nonumber \\
  \tau_m &= \frac{K_t}{R} V - \frac{K_t}{K_v R} \omega_m
  \intertext{Since $\tau_m G = \tau_{arm}$ and $\omega_m = G \omega_{arm}$,}
  \left(\frac{\tau_{arm}}{G}\right) &= \frac{K_t}{R} V -
    \frac{K_t}{K_v R} (G \omega_{arm}) \nonumber \\
  \frac{\tau_{arm}}{G} &= \frac{K_t}{R} V - \frac{G K_t}{K_v R} \omega_{arm}
    \nonumber \\
  \tau_{arm} &= \frac{G K_t}{R} V - \frac{G^2 K_t}{K_v R} \omega_{arm}
    \label{eq:tau_arm}
  \intertext{The torque applied to the arm is defined as}
  \tau_{arm} &= J \dot{\omega}_{arm} \label{eq:tau_arm_def}
  \intertext{where $J$ is the moment of inertia of the arm and
    $\dot{\omega}_{arm}$ is the angular acceleration. Substitute equation
    \eqref{eq:tau_arm_def} into equation \eqref{eq:tau_arm}.}
  (J \dot{\omega}_{arm}) &= \frac{G K_t}{R} V - \frac{G^2 K_t}{K_v R}
    \omega_{arm} \nonumber \\
  \dot{\omega}_{arm} &= -\frac{G^2 K_t}{K_v RJ} \omega_{arm} +
    \frac{G K_t}{RJ} V \nonumber
  \intertext{We'll relabel $\omega_{arm}$ as $\omega$ for convenience.}
  \dot{\omega} &= -\frac{G^2 K_t}{K_v RJ} \omega + \frac{G K_t}{RJ} V
    \label{eq:dot_omega_arm}
\end{align}

This model will be converted to state-space notation in section
\ref{sec:ss_model_single-jointed_arm}.

\subsection{Calculating constants}

\subsubsection{Moment of inertia J}

Given the simplicity of this mechanism, it may be easier to compute this value
theoretically using material properties in CAD. $J$ can also be approximated as
the moment of inertia of a thin rod rotating around one end. Therefore
\begin{equation}
  J = \frac{1}{3}ml^2
\end{equation}

where $m$ is the mass of the arm and $l$ is the length of the arm. Otherwise, a
procedure for measuring it experimentally is presented below.

First, rearrange equation \eqref{eq:dot_omega_arm} into the form $y = mx + b$
such that $J$ is in the numerator of $m$.
\begin{align}
  \dot{\omega} &= -\frac{G^2 K_t}{K_v RJ} \omega + \frac{G K_t}{RJ} V \nonumber
    \\
  J\dot{\omega} &= -\frac{G^2 K_t}{K_v R} \omega + \frac{G K_t}{R} V \nonumber
  \intertext{Multiply by $\frac{K_v R}{G^2 K_t}$ on both sides.}
  \frac{J K_v R}{G^2 K_t} \dot{\omega} &= -\omega + \frac{G K_t}{R} \cdot
    \frac{K_v R}{G^2 K_t} V \nonumber \\
  \frac{J K_v R}{G^2 K_t} \dot{\omega} &= -\omega + \frac{K_v}{G} V \nonumber \\
  \omega &= -\frac{J K_v R}{G^2 K_t} \dot{\omega} + \frac{K_v}{G} V
    \label{eq:arm_J_regression}
\end{align}

The test procedure is as follows.
\begin{enumerate}
  \item Orient the arm such that its axis of rotation is aligned with gravity
    (i.e., the arm is on its side). This avoids gravity affecting the
    measurements.
  \item Run the arm forward at a constant voltage. Record the angular velocity
    over time.
  \item Compute the angular acceleration from the angular velocity data as the
    difference between each sample divided by the time between them.
  \item Perform a linear regression of angular velocity versus angular
    acceleration. The slope of this line has the form $-\frac{J K_v R}{G^2 K_t}$
    as per equation \eqref{eq:arm_J_regression}.
  \item Multiply the slope by $-\frac{G^2 K_t}{K_v R}$ to obtain a least squares
    estimate of $J$.
\end{enumerate}

\input{\chapterpath/pendulum}
\section{Differential drive}
\label{sec:ss_model_differential_drive}

This drivetrain consists of two DC motors per side which are chained together on
their respective sides and drive wheels which are assumed to be massless.
\begin{bookfigure}
  \begin{bookminifig}{2}
    \input{figs/differential-drive-fbd}
    \caption{Differential drive dimensions}
  \end{bookminifig}
  \begin{bookminifig}{2}
    \input{figs/differential-drive-system-diagram}
    \caption{Differential drive coordinate frame}
  \end{bookminifig}
\end{bookfigure}

\subsection{Velocity subspace state-space model}
\index{FRC models!differential drive equations}

By equations \eqref{eq:diff_drive_model_right} and
\eqref{eq:diff_drive_model_left}
\begin{align*}
  \dot{v}_l &= \left(\frac{1}{m} + \frac{r_b^2}{J}\right)
    \left(C_1 v_l + C_2 V_l\right) +
    \left(\frac{1}{m} - \frac{r_b^2}{J}\right) \left(C_3 v_r + C_4 V_r\right) \\
  \dot{v}_r &= \left(\frac{1}{m} - \frac{r_b^2}{J}\right)
    \left(C_1 v_l + C_2 V_l\right) +
    \left(\frac{1}{m} + \frac{r_b^2}{J}\right) \left(C_3 v_r + C_4 V_r\right)
  \intertext{Regroup the terms into states $v_l$ and $v_r$ and inputs $V_l$ and
    $V_r$.}
  \dot{v}_l &= \left(\frac{1}{m} + \frac{r_b^2}{J}\right) C_1 v_l +
    \left(\frac{1}{m} + \frac{r_b^2}{J}\right) C_2 V_l \\
  &\qquad + \left(\frac{1}{m} - \frac{r_b^2}{J}\right) C_3 v_r +
    \left(\frac{1}{m} - \frac{r_b^2}{J}\right) C_4 V_r \\
  \dot{v}_r &= \left(\frac{1}{m} - \frac{r_b^2}{J}\right) C_1 v_l +
    \left(\frac{1}{m} - \frac{r_b^2}{J}\right) C_2 V_l \\
  &\qquad + \left(\frac{1}{m} + \frac{r_b^2}{J}\right) C_3 v_r +
    \left(\frac{1}{m} + \frac{r_b^2}{J}\right) C_4 V_r
  \shortintertext{}
  \dot{v}_l &= \left(\frac{1}{m} + \frac{r_b^2}{J}\right) C_1 v_l +
    \left(\frac{1}{m} - \frac{r_b^2}{J}\right) C_3 v_r \\
  &\qquad + \left(\frac{1}{m} + \frac{r_b^2}{J}\right) C_2 V_l +
    \left(\frac{1}{m} - \frac{r_b^2}{J}\right) C_4 V_r \\
  \dot{v}_r &= \left(\frac{1}{m} - \frac{r_b^2}{J}\right) C_1 v_l +
    \left(\frac{1}{m} + \frac{r_b^2}{J}\right) C_3 v_r \\
  &\qquad + \left(\frac{1}{m} - \frac{r_b^2}{J}\right) C_2 V_l +
    \left(\frac{1}{m} + \frac{r_b^2}{J}\right) C_4 V_r
  \intertext{Factor out $v_l$ and $v_r$ into a column vector and $V_l$ and $V_r$
    into a column vector.}
  \dot{\begin{bmatrix}
    v_l \\
    v_r
  \end{bmatrix}} &=
  \begin{bmatrix}
    \left(\frac{1}{m} + \frac{r_b^2}{J}\right) C_1 &
    \left(\frac{1}{m} - \frac{r_b^2}{J}\right) C_3 \\
    \left(\frac{1}{m} - \frac{r_b^2}{J}\right) C_1 &
    \left(\frac{1}{m} + \frac{r_b^2}{J}\right) C_3
  \end{bmatrix}
  \begin{bmatrix}
    v_l \\
    v_r
  \end{bmatrix} \\
  &\qquad +
  \begin{bmatrix}
    \left(\frac{1}{m} + \frac{r_b^2}{J}\right) C_2 &
    \left(\frac{1}{m} - \frac{r_b^2}{J}\right) C_4 \\
    \left(\frac{1}{m} - \frac{r_b^2}{J}\right) C_2 &
    \left(\frac{1}{m} + \frac{r_b^2}{J}\right) C_4
  \end{bmatrix}
  \begin{bmatrix}
    V_l \\
    V_r
  \end{bmatrix}
\end{align*}
\begin{theorem}[Differential drive velocity state-space model]
  \begin{equation*}
    \dot{\mat{x}} = \mat{A} \mat{x} + \mat{B} \mat{u}
  \end{equation*}
  \begin{equation*}
    \mat{x} =
    \begin{bmatrix}
      v_l \\
      v_r
    \end{bmatrix} =
    \begin{bmatrix}
      \text{left velocity} \\
      \text{right velocity}
    \end{bmatrix}
    \quad
    \mat{u} =
    \begin{bmatrix}
      V_l \\
      V_r
    \end{bmatrix} =
    \begin{bmatrix}
      \text{left voltage} \\
      \text{right voltage}
    \end{bmatrix}
  \end{equation*}
  \begin{align}
    \mat{A} &=
    \begin{bmatrix}
      \left(\frac{1}{m} + \frac{r_b^2}{J}\right) C_1 & \left(\frac{1}{m} - \frac{r_b^2}{J}\right) C_3 \\
      \left(\frac{1}{m} - \frac{r_b^2}{J}\right) C_1 & \left(\frac{1}{m} + \frac{r_b^2}{J}\right) C_3
    \end{bmatrix} \\
    \mat{B} &=
    \begin{bmatrix}
      \left(\frac{1}{m} + \frac{r_b^2}{J}\right) C_2 & \left(\frac{1}{m} - \frac{r_b^2}{J}\right) C_4 \\
      \left(\frac{1}{m} - \frac{r_b^2}{J}\right) C_2 & \left(\frac{1}{m} + \frac{r_b^2}{J}\right) C_4
    \end{bmatrix}
  \end{align}

  where $C_1 = -\frac{G_l^2 K_t}{K_v R r^2}$, $C_2 = \frac{G_l K_t}{Rr}$,
  $C_3 = -\frac{G_r^2 K_t}{K_v R r^2}$, and $C_4 = \frac{G_r K_t}{Rr}$.
\end{theorem}

\subsubsection{Simulation}

Python Control will be used to \glslink{discretization}{discretize} the
\gls{model} and simulate it. One of the frccontrol
examples\footnote{\url{https://github.com/calcmogul/frccontrol/blob/main/examples/differential_drive.py}}
creates and tests a controller for it. Figure \ref{fig:diff_drive_response}
shows the closed-loop \gls{system} response.
\begin{svg}{build/\chapterpath/differential_drive_response}
  \caption{Drivetrain response}
  \label{fig:diff_drive_response}
\end{svg}

Given the high inertia in drivetrains, it's better to drive the \gls{reference}
with a motion profile instead of a \gls{step input} for reproducibility.

\subsection{Heading state-space model}

We can control heading by augmenting the state with that. The change in heading
is defined as
\begin{equation*}
  \dot{\theta} = \frac{v_r - v_l}{2r_b} = \frac{v_r}{2r_b} - \frac{v_l}{2r_b}
\end{equation*}

This gives the following linear model.
\begin{theorem}[Differential drive heading state-space model]
  \begin{equation*}
    \dot{\mat{x}} = \mat{A}\mat{x} + \mat{B}\mat{u}
  \end{equation*}
  \begin{equation*}
    \mat{x} =
    \begin{bmatrix}
      \theta \\
      v_l \\
      v_r
    \end{bmatrix} =
    \begin{bmatrix}
      \text{heading} \\
      \text{left velocity} \\
      \text{right velocity}
    \end{bmatrix}
    \quad
    \mat{u} =
    \begin{bmatrix}
      V_l \\
      V_r
    \end{bmatrix} =
    \begin{bmatrix}
      \text{left voltage} \\
      \text{right voltage}
    \end{bmatrix}
  \end{equation*}
  \begin{align}
    \mat{A} &=
    \begin{bmatrix}
      0 & -\frac{1}{2r_b} & \frac{1}{2r_b} \\
      0 & \left(\frac{1}{m} + \frac{r_b^2}{J}\right) C_1 &
        \left(\frac{1}{m} - \frac{r_b^2}{J}\right) C_3 \\
      0 & \left(\frac{1}{m} - \frac{r_b^2}{J}\right) C_1 &
        \left(\frac{1}{m} + \frac{r_b^2}{J}\right) C_3
    \end{bmatrix} \\
    \mat{B} &=
    \begin{bmatrix}
      0 & 0 \\
      \left(\frac{1}{m} + \frac{r_b^2}{J}\right) C_2 &
      \left(\frac{1}{m} - \frac{r_b^2}{J}\right) C_4 \\
      \left(\frac{1}{m} - \frac{r_b^2}{J}\right) C_2 &
      \left(\frac{1}{m} + \frac{r_b^2}{J}\right) C_4
    \end{bmatrix}
  \end{align}

  where $C_1 = -\frac{G_l^2 K_t}{K_v R r^2}$, $C_2 = \frac{G_l K_t}{Rr}$,
  $C_3 = -\frac{G_r^2 K_t}{K_v R r^2}$, and $C_4 = \frac{G_r K_t}{Rr}$. The
  constants $C_1$ through $C_4$ are from the derivation in section
  \ref{sec:differential_drive}.
\end{theorem}

The velocity states are required to make the heading controllable.

\subsection{Linear time-varying model}
\index{controller design!linear time-varying control}
\index{nonlinear control!linear time-varying control}
\index{optimal control!linear time-varying control}

We can control the drivetrain's global pose $(x, y, \theta)$ by augmenting the
state with $x$ and $y$. The change in global pose is defined by these three
equations.
\begin{align*}
  \dot{x} &= \frac{v_l + v_r}{2}\cos\theta = \frac{v_r}{2}\cos\theta +
    \frac{v_l}{2}\cos\theta \\
  \dot{y} &= \frac{v_l + v_r}{2}\sin\theta = \frac{v_r}{2}\sin\theta +
    \frac{v_l}{2}\sin\theta \\
  \dot{\theta} &= \frac{v_r - v_l}{2r_b} = \frac{v_r}{2r_b} - \frac{v_l}{2r_b}
\end{align*}

This augmented model is a nonlinear vector function where
$\mat{x} = \begin{bmatrix} x & y & \theta & v_l & v_r \end{bmatrix}\T$ and
$\mat{u} = \begin{bmatrix} V_l & V_r \end{bmatrix}\T$.
\begin{align}
  &f(\mat{x}, \mat{u}) = \nonumber \\
  &\qquad \begin{bmatrix}
    \frac{v_r}{2}\cos\theta + \frac{v_l}{2}\cos\theta \\
    \frac{v_r}{2}\sin\theta + \frac{v_l}{2}\sin\theta \\
    \frac{v_r}{2r_b} - \frac{v_l}{2r_b} \\
    \left(\frac{1}{m} + \frac{r_b^2}{J}\right) C_1 v_l +
      \left(\frac{1}{m} - \frac{r_b^2}{J}\right) C_3 v_r +
      \left(\frac{1}{m} + \frac{r_b^2}{J}\right) C_2 V_l +
      \left(\frac{1}{m} - \frac{r_b^2}{J}\right) C_4 V_r \\
    \left(\frac{1}{m} - \frac{r_b^2}{J}\right) C_1 v_l +
      \left(\frac{1}{m} + \frac{r_b^2}{J}\right) C_3 v_r +
      \left(\frac{1}{m} - \frac{r_b^2}{J}\right) C_2 V_l +
      \left(\frac{1}{m} + \frac{r_b^2}{J}\right) C_4 V_r
  \end{bmatrix}
  \label{eq:ltv_diff_drive_f}
\end{align}

As mentioned in chapter \ref{ch:nonlinear_control}, one can approximate a
nonlinear system via linearizations around points of interest in the state-space
and design controllers for those linearized subspaces. If we sample
linearization points progressively closer together, we converge on a control
policy for the original nonlinear system. Since the linear \gls{plant} being
controlled varies with time, its controller is called a linear time-varying
(LTV) controller.

If we use LQRs for the linearized subspaces, the nonlinear control policy will
also be locally optimal. We'll be taking this approach with a differential
drive. To create an LQR, we need to linearize equation
\eqref{eq:ltv_diff_drive_f}.
\begin{align*}
  \frac{\partial f(\mat{x}, \mat{u})}{\partial\mat{x}} &=
  \begin{bmatrix}
    0 & 0 & -\frac{v_l + v_r}{2}\sin\theta & \frac{1}{2}\cos\theta &
      \frac{1}{2}\cos\theta \\
    0 & 0 & \frac{v_l + v_r}{2}\cos\theta & \frac{1}{2}\sin\theta &
      \frac{1}{2}\sin\theta \\
    0 & 0 & 0 & -\frac{1}{2r_b} & \frac{1}{2r_b} \\
    0 & 0 & 0 & \left(\frac{1}{m} + \frac{r_b^2}{J}\right) C_1 &
      \left(\frac{1}{m} - \frac{r_b^2}{J}\right) C_3 \\
    0 & 0 & 0 & \left(\frac{1}{m} - \frac{r_b^2}{J}\right) C_1 &
      \left(\frac{1}{m} + \frac{r_b^2}{J}\right) C_3
  \end{bmatrix} \\
  \frac{\partial f(\mat{x}, \mat{u})}{\partial\mat{u}} &=
  \begin{bmatrix}
    0 & 0 \\
    0 & 0 \\
    0 & 0 \\
    \left(\frac{1}{m} + \frac{r_b^2}{J}\right) C_2 &
    \left(\frac{1}{m} - \frac{r_b^2}{J}\right) C_4 \\
    \left(\frac{1}{m} - \frac{r_b^2}{J}\right) C_2 &
    \left(\frac{1}{m} + \frac{r_b^2}{J}\right) C_4
  \end{bmatrix}
\end{align*}

Therefore,
\begin{theorem}[Linear time-varying differential drive state-space model]
  \begin{equation*}
    \dot{\mat{x}} = \mat{A}\mat{x} + \mat{B}\mat{u}
  \end{equation*}
  \begin{equation*}
    \mat{x} =
    \begin{bmatrix}
      x \\
      y \\
      \theta \\
      v_l \\
      v_r
    \end{bmatrix} =
    \begin{bmatrix}
      \text{x position} \\
      \text{y position} \\
      \text{heading} \\
      \text{left velocity} \\
      \text{right velocity}
    \end{bmatrix}
    \quad
    \mat{u} =
    \begin{bmatrix}
      V_l \\
      V_r
    \end{bmatrix} =
    \begin{bmatrix}
      \text{left voltage} \\
      \text{right voltage}
    \end{bmatrix}
  \end{equation*}
  \begin{align}
    \mat{A} &=
    \begin{bmatrix}
      0 & 0 & -vs & \frac{1}{2}c & \frac{1}{2}c \\
      0 & 0 & vc & \frac{1}{2}s & \frac{1}{2}s \\
      0 & 0 & 0 & -\frac{1}{2r_b} & \frac{1}{2r_b} \\
      0 & 0 & 0 & \left(\frac{1}{m} + \frac{r_b^2}{J}\right) C_1 &
        \left(\frac{1}{m} - \frac{r_b^2}{J}\right) C_3 \\
      0 & 0 & 0 & \left(\frac{1}{m} - \frac{r_b^2}{J}\right) C_1 &
        \left(\frac{1}{m} + \frac{r_b^2}{J}\right) C_3
    \end{bmatrix} \\
    \mat{B} &=
    \begin{bmatrix}
      0 & 0 \\
      0 & 0 \\
      0 & 0 \\
      \left(\frac{1}{m} + \frac{r_b^2}{J}\right) C_2 &
      \left(\frac{1}{m} - \frac{r_b^2}{J}\right) C_4 \\
      \left(\frac{1}{m} - \frac{r_b^2}{J}\right) C_2 &
      \left(\frac{1}{m} + \frac{r_b^2}{J}\right) C_4
    \end{bmatrix}
  \end{align}

  where $v = \frac{v_l + v_r}{2}$, $c = \cos\theta$, $s = \sin\theta$,
  $C_1 = -\frac{G_l^2 K_t}{K_v R r^2}$, $C_2 = \frac{G_l K_t}{Rr}$,
  $C_3 = -\frac{G_r^2 K_t}{K_v R r^2}$, and $C_4 = \frac{G_r K_t}{Rr}$. The
  constants $C_1$ through $C_4$ are from the derivation in section
  \ref{sec:differential_drive}.
\end{theorem}

We can also use this in an extended Kalman filter as is since the measurement
model ($\mat{y} = \mat{C}\mat{x} + \mat{D}\mat{u}$) is linear.

\subsection{Improving model accuracy}

Figures \ref{fig:ltv_diff_drive_nonrotated_firstorder_xy} and
\ref{fig:ltv_diff_drive_nonrotated_firstorder_response} demonstrate the
tracking behavior of the linearized differential drive controller.
\begin{bookfigure}
  \begin{minisvg}{2}{build/\chapterpath/ltv_diff_drive_nonrotated_firstorder_xy}
    \caption{Linear time-varying differential drive controller x-y plot
      (first-order)}
    \label{fig:ltv_diff_drive_nonrotated_firstorder_xy}
  \end{minisvg}
  \hfill
  \begin{minisvg}{2}{build/\chapterpath/ltv_diff_drive_nonrotated_firstorder_response}
    \caption{Linear time-varying differential drive controller response
      (first-order)}
    \label{fig:ltv_diff_drive_nonrotated_firstorder_response}
  \end{minisvg}
\end{bookfigure}

The linearized differential drive model doesn't track well because the
first-order linearization of $\mat{A}$ doesn't capture the full heading
dynamics, making the \gls{model} update inaccurate. This linearization
inaccuracy is evident in the Hessian matrix (second partial derivative with
respect to the state vector) being nonzero.
\begin{equation*}
  \frac{\partial^2 f(\mat{x}, \mat{u})}{\partial\mat{x}^2} =
  \begin{bmatrix}
    0 & 0 & -\frac{v_l + v_r}{2}\cos\theta & 0 & 0 \\
    0 & 0 & -\frac{v_l + v_r}{2}\sin\theta & 0 & 0 \\
    0 & 0 & 0 & 0 & 0 \\
    0 & 0 & 0 & 0 & 0 \\
    0 & 0 & 0 & 0 & 0
  \end{bmatrix}
\end{equation*}

The second-order Taylor series expansion of the \gls{model} around $\mat{x}_0$
would be
\begin{equation*}
  f(\mat{x}, \mat{u}_0) \approx f(\mat{x}_0, \mat{u}_0) +
    \frac{\partial f(\mat{x}, \mat{u})}{\partial\mat{x}}(\mat{x} - \mat{x}_0) +
    \frac{1}{2}\frac{\partial^2 f(\mat{x}, \mat{u})}{\partial\mat{x}^2}
    (\mat{x} - \mat{x}_0)^2
\end{equation*}

To include higher-order dynamics in the linearized differential drive model
integration, we'll apply the Dormand-Prince integration method (RKDP) from
theorem \ref{thm:rkdp} to equation \eqref{eq:ltv_diff_drive_f}.

Figures \ref{fig:ltv_diff_drive_nonrotated_xy} and
\ref{fig:ltv_diff_drive_nonrotated_response} show a simulation using RKDP
instead of the first-order \gls{model}.
\begin{bookfigure}
  \begin{minisvg}{2}{build/\chapterpath/ltv_diff_drive_nonrotated_xy}
    \caption{Linear time-varying differential drive controller (global reference
        frame formulation) x-y plot}
    \label{fig:ltv_diff_drive_nonrotated_xy}
  \end{minisvg}
  \hfill
  \begin{minisvg}{2}{build/\chapterpath/ltv_diff_drive_nonrotated_response}
    \caption{Linear time-varying differential drive controller (global reference
        frame formulation) response}
    \label{fig:ltv_diff_drive_nonrotated_response}
  \end{minisvg}
\end{bookfigure}

\subsection{Cross track error controller}

Figures \ref{fig:ltv_diff_drive_nonrotated_xy} and
\ref{fig:ltv_diff_drive_nonrotated_response} show the tracking performance of
the linearized differential drive controller for a given trajectory. The
performance-effort trade-off can be tuned rather intuitively via the Q and R
gains. However, if the $x$ and $y$ error cost are too high, the $x$ and $y$
components of the controller will fight each other, and it will take longer to
converge to the path. This can be fixed by applying a clockwise rotation matrix
to the global tracking error to transform it into the robot's coordinate frame.
\begin{equation*}
  \crdfrm{R}{\begin{bmatrix}
    e_x \\
    e_y \\
    e_\theta
  \end{bmatrix}} =
  \begin{bmatrix}
    \cos\theta & \sin\theta & 0 \\
    -\sin\theta & \cos\theta & 0 \\
    0 & 0 & 1
  \end{bmatrix}
  \crdfrm{G}{\begin{bmatrix}
    e_x \\
    e_y \\
    e_\theta
  \end{bmatrix}}
\end{equation*}

where the the superscript $R$ represents the robot's coordinate frame and the
superscript $G$ represents the global coordinate frame.

With this transformation, the $x$ and $y$ error cost in LQR penalize the error
ahead of the robot and cross-track error respectively instead of global pose
error. Since the cross-track error is always measured from the robot's
coordinate frame, the \gls{model} used to compute the LQR should be linearized
around $\theta = 0$ at all times.
\begin{align*}
  \mat{A} &=
  \begin{bmatrix}
    0 & 0 & -\frac{v_l + v_r}{2}\sin 0 & \frac{1}{2}\cos 0 &
      \frac{1}{2}\cos 0 \\
    0 & 0 & \frac{v_l + v_r}{2}\cos 0 & \frac{1}{2}\sin 0 &
      \frac{1}{2}\sin 0 \\
    0 & 0 & 0 & -\frac{1}{2r_b} & \frac{1}{2r_b} \\
    0 & 0 & 0 & \left(\frac{1}{m} + \frac{r_b^2}{J}\right) C_1 &
      \left(\frac{1}{m} - \frac{r_b^2}{J}\right) C_3 \\
    0 & 0 & 0 & \left(\frac{1}{m} - \frac{r_b^2}{J}\right) C_1 &
      \left(\frac{1}{m} + \frac{r_b^2}{J}\right) C_3
  \end{bmatrix} \\
  \mat{A} &=
  \begin{bmatrix}
    0 & 0 & 0 & \frac{1}{2} & \frac{1}{2} \\
    0 & 0 & \frac{v_l + v_r}{2} & 0 & 0 \\
    0 & 0 & 0 & -\frac{1}{2r_b} & \frac{1}{2r_b} \\
    0 & 0 & 0 & \left(\frac{1}{m} + \frac{r_b^2}{J}\right) C_1 &
      \left(\frac{1}{m} - \frac{r_b^2}{J}\right) C_3 \\
    0 & 0 & 0 & \left(\frac{1}{m} - \frac{r_b^2}{J}\right) C_1 &
      \left(\frac{1}{m} + \frac{r_b^2}{J}\right) C_3
  \end{bmatrix}
\end{align*}
\begin{theorem}[Linear time-varying differential drive controller]
  \label{thm:linear_time-varying_diff_drive_controller}

  Let the differential drive dynamics be of the form
  $\dot{\mat{x}} = f(\mat{x}) + \mat{B}\mat{u}$ where
  \begin{equation*}
    \mat{x} =
    \begin{bmatrix}
      x \\
      y \\
      \theta \\
      v_l \\
      v_r
    \end{bmatrix} =
    \begin{bmatrix}
      \text{x position} \\
      \text{y position} \\
      \text{heading} \\
      \text{left velocity} \\
      \text{right velocity}
    \end{bmatrix}
    \quad
    \mat{u} =
    \begin{bmatrix}
      V_l \\
      V_r
    \end{bmatrix} =
    \begin{bmatrix}
      \text{left voltage} \\
      \text{right voltage}
    \end{bmatrix}
  \end{equation*}
  \begin{align}
    \mat{A} &=
    \left.\frac{\partial f(\mat{x})}{\partial\mat{x}}\right|_{\theta = 0} =
    \begin{bmatrix}
      0 & 0 & 0 & \frac{1}{2} & \frac{1}{2} \\
      0 & 0 & v & 0 & 0 \\
      0 & 0 & 0 & -\frac{1}{2r_b} & \frac{1}{2r_b} \\
      0 & 0 & 0 & \left(\frac{1}{m} + \frac{r_b^2}{J}\right) C_1 &
        \left(\frac{1}{m} - \frac{r_b^2}{J}\right) C_3 \\
      0 & 0 & 0 & \left(\frac{1}{m} - \frac{r_b^2}{J}\right) C_1 &
        \left(\frac{1}{m} + \frac{r_b^2}{J}\right) C_3
    \end{bmatrix} \\
    \mat{B} &=
    \begin{bmatrix}
      0 & 0 \\
      0 & 0 \\
      0 & 0 \\
      \left(\frac{1}{m} + \frac{r_b^2}{J}\right) C_2 &
      \left(\frac{1}{m} - \frac{r_b^2}{J}\right) C_4 \\
      \left(\frac{1}{m} - \frac{r_b^2}{J}\right) C_2 &
      \left(\frac{1}{m} + \frac{r_b^2}{J}\right) C_4
    \end{bmatrix}
  \end{align}

  where $v = \frac{v_l + v_r}{2}$, $C_1 = -\frac{G_l^2 K_t}{K_v R r^2}$,
  $C_2 = \frac{G_l K_t}{Rr}$, $C_3 = -\frac{G_r^2 K_t}{K_v R r^2}$, and
  $C_4 = \frac{G_r K_t}{Rr}$. The constants $C_1$ through $C_4$ are from the
  derivation in section \ref{sec:differential_drive}.

  The linear time-varying differential drive controller is
  \begin{equation}
    \mat{u} = \mat{K}
    \left[
      \begin{array}{c|c}
        \begin{array}{cc}
          \cos\theta & \sin\theta \\
          -\sin\theta & \cos\theta
        \end{array} & \mat{0}_{2 \times 3} \\
        \hline
        \mat{0}_{3 \times 2} & \mat{I}_{3 \times 3}
      \end{array}
    \right]
    (\mat{r} - \mat{x})
  \end{equation}

  At each timestep, the LQR controller gain $\mat{K}$ is computed for the
  $(\mat{A}, \mat{B})$ pair evaluated at the current state.
\end{theorem}

With the \gls{model} in theorem
\ref{thm:linear_time-varying_diff_drive_controller}, $y$ is uncontrollable at
$v = 0$ because the row corresponding to $y$ becomes the zero vector. This means
the state dynamics and inputs can no longer affect $y$. This is obvious given
that nonholonomic drivetrains can't move sideways. Some DARE solvers throw
errors in this case, but one can avoid it by linearizing the model around a
slightly nonzero velocity instead.

The controller in theorem \ref{thm:linear_time-varying_diff_drive_controller}
results in figures \ref{fig:ltv_diff_drive_traj_xy} and
\ref{fig:ltv_diff_drive_traj_response}, which show slightly better tracking
performance than the previous formulation.
\begin{bookfigure}
  \begin{minisvg}{2}{build/\chapterpath/ltv_diff_drive_traj_xy}
    \caption{Linear time-varying differential drive controller x-y plot}
    \label{fig:ltv_diff_drive_traj_xy}
  \end{minisvg}
  \hfill
  \begin{minisvg}{2}{build/\chapterpath/ltv_diff_drive_traj_response}
    \caption{Linear time-varying differential drive controller response}
    \label{fig:ltv_diff_drive_traj_response}
  \end{minisvg}
\end{bookfigure}

\subsection{Nonlinear observer design}

\subsubsection{Encoder position augmentation}

Estimation of the global pose can be significantly improved if encoder position
measurements are used instead of velocity measurements. By augmenting the plant
with the line integral of each wheel's velocity over time, we can provide a
mapping from model states to position measurements. We can augment the linear
subspace of the model as follows.

Augment the matrix equation with position states $x_l$ and $x_r$, which have the
model equations $\dot{x}_l = v_l$ and $\dot{x}_r = v_r$. The matrix elements
corresponding to $v_l$ in the first equation and $v_r$ in the second equation
will be $1$, and the others will be $0$ since they don't appear, so
$\dot{x}_l = 1v_l + 0v_r + 0x_l + 0x_r + 0V_l + 0V_r$ and
$\dot{x}_r = 0v_l + 1v_r + 0x_l + 0x_r + 0V_l + 0V_r$. The existing rows will
have zeroes inserted where $x_l$ and $x_r$ are multiplied in.
\begin{equation*}
  \dot{\begin{bmatrix}
    x_l \\
    x_r
  \end{bmatrix}} =
  \begin{bmatrix}
    1 & 0 \\
    0 & 1
  \end{bmatrix}
  \begin{bmatrix}
    v_l \\
    v_r
  \end{bmatrix} +
  \begin{bmatrix}
    0 & 0 \\
    0 & 0
  \end{bmatrix}
  \begin{bmatrix}
    V_l \\
    V_r
  \end{bmatrix}
\end{equation*}

This produces the following linear subspace over
$\mat{x} = \begin{bmatrix}v_l & v_r & x_l & x_r\end{bmatrix}\T$.
\begin{align}
  \mat{A} &=
  \begin{bmatrix}
    \left(\frac{1}{m} + \frac{r_b^2}{J}\right) C_1 &
      \left(\frac{1}{m} - \frac{r_b^2}{J}\right) C_3 & 0 & 0 \\
    \left(\frac{1}{m} - \frac{r_b^2}{J}\right) C_1 &
      \left(\frac{1}{m} + \frac{r_b^2}{J}\right) C_3 & 0 & 0 \\
    1 & 0 & 0 & 0 \\
    0 & 1 & 0 & 0
  \end{bmatrix} \\
  \mat{B} &=
  \begin{bmatrix}
    \left(\frac{1}{m} + \frac{r_b^2}{J}\right) C_2 &
      \left(\frac{1}{m} - \frac{r_b^2}{J}\right) C_4 \\
    \left(\frac{1}{m} - \frac{r_b^2}{J}\right) C_2 &
      \left(\frac{1}{m} + \frac{r_b^2}{J}\right) C_4 \\
    0 & 0 \\
    0 & 0
  \end{bmatrix}
  \label{eq:diff_drive_linear_subspace}
\end{align}

The measurement model for the complete nonlinear model is now
$\mat{y} = \begin{bmatrix}\theta & x_l & x_r\end{bmatrix}\T$ instead of
$\mat{y} = \begin{bmatrix}\theta & v_l & v_r\end{bmatrix}\T$.

\subsubsection{U error estimation}

As per subsection \ref{subsec:input_error_estimation}, we will now augment the
\gls{model} so $u_{error}$ states are added to the \glspl{control input}.

The \gls{plant} and \gls{observer} augmentations should be performed before the
\gls{model} is \glslink{discretization}{discretized}. After the \gls{controller}
gain is computed with the unaugmented discrete \gls{model}, the controller may
be augmented. Therefore, the \gls{plant} and \gls{observer} augmentations assume
a continuous \gls{model} and the \gls{controller} augmentation assumes a
discrete \gls{controller}.

The three $u_{error}$ states we'll be adding are $u_{error,l}$, $u_{error,r}$,
and $u_{error,heading}$ for left voltage error, right voltage error, and heading
error respectively. The left and right wheel positions are filtered encoder
positions and are not adjusted for heading error. The turning angle computed
from the left and right wheel positions is adjusted by the gyroscope heading.
The heading $u_{error}$ state is the heading error between what the wheel
positions imply and the gyroscope measurement.

The full state is thus
\begin{equation*}
  \mat{x} =
  \begin{bmatrix}
    x \\
    y \\
    \theta \\
    v_l \\
    v_r \\
    x_l \\
    x_r \\
    u_{error,l} \\
    u_{error,r} \\
    u_{error,heading}
  \end{bmatrix}
\end{equation*}

The complete nonlinear model is as follows. Let $v = \frac{v_l + v_r}{2}$. The
three $u_{error}$ states augment the linear subspace, so the nonlinear pose
dynamics are the same.
\begin{align}
  \dot{\begin{bmatrix}
    x \\
    y \\
    \theta
  \end{bmatrix}} &=
    \begin{bmatrix}
      v\cos\theta \\
      v\sin\theta \\
      \frac{v_r}{2r_b} - \frac{v_l}{2r_b}
    \end{bmatrix}
\end{align}

The left and right voltage error states should be mapped to the corresponding
velocity states, so the system matrix should be augmented with $\mat{B}$.

The heading $u_{error}$ is measuring counterclockwise encoder understeer
relative to the gyroscope heading, so it should add to the left position and
subtract from the right position for clockwise correction of encoder positions.
That corresponds to the following input mapping vector.
\begin{equation*}
  \mat{B}_{\theta} = \begin{bmatrix}
    0 \\
    0 \\
    1 \\
    -1
  \end{bmatrix}
\end{equation*}

Now we'll augment the linear system matrix horizontally to accomodate the
$u_{error}$ states.
\begin{equation}
  \dot{\begin{bmatrix}
    v_l \\
    v_r \\
    x_l \\
    x_r
  \end{bmatrix}} =
    \begin{bmatrix}
      \mat{A} & \mat{B} & \mat{B}_{\theta}
    \end{bmatrix}
    \begin{bmatrix}
      v_l \\
      v_r \\
      x_l \\
      x_r \\
      u_{error,l} \\
      u_{error,r} \\
      u_{error,heading}
    \end{bmatrix} + \mat{B}\mat{u}
\end{equation}

$\mat{A}$ and $\mat{B}$ are the linear subspace from equation
\eqref{eq:diff_drive_linear_subspace}.

The $u_{error}$ states have no dynamics. The observer selects them to minimize
the difference between the expected and actual measurements.
\begin{equation}
  \dot{\begin{bmatrix}
    u_{error,l} \\
    u_{error,r} \\
    u_{error,heading}
  \end{bmatrix}} = \mat{0}_{3 \times 1}
\end{equation}

The controller is augmented as follows.
\begin{equation}
  \mat{K}_{error} =
  \begin{bmatrix}
    1 & 0 & 0 \\
    0 & 1 & 0
  \end{bmatrix}
  \quad
  \mat{K}_{aug} = \begin{bmatrix}
    \mat{K} & \mat{K}_{error}
  \end{bmatrix}
  \quad
  \mat{r}_{aug} = \begin{bmatrix}
    \mat{r} \\
    0 \\
    0 \\
    0
  \end{bmatrix}
\end{equation}

This controller augmentation compensates for unmodeled dynamics like:
\begin{enumerate}
  \item Understeer caused by wheel friction inherent in skid-steer robots
  \item Battery voltage drop under load, which reduces the available control
    authority
\end{enumerate}
\begin{remark}
  The process noise for the voltage error states should be how much the voltage
  can be expected to drop. The heading error state should be the encoder
  \gls{model} uncertainty.
\end{remark}

