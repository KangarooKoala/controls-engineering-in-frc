\section{Flywheel}

This flywheel consists of a DC motor attached to a spinning mass of
non-negligible moment of inertia.
\begin{bookfigure}
  \input{figs/flywheel-system-diagram}
  \caption{Flywheel system diagram}
  \label{fig:flywheel}
\end{bookfigure}

Gear ratios are written as output over input, so $G$ is greater than one in
figure \ref{fig:flywheel}.

\subsection{Equations of motion}

We want to derive an equation for the flywheel angular acceleration
$\dot{\omega}_f$ given an input voltage $V$, which we can integrate to get
flywheel angular velocity.

We will start with the equation derived earlier for a DC motor, equation
\eqref{eq:motor_tau_V}.
\begin{align}
  V &= \frac{\tau_m}{K_t} R + \frac{\omega_m}{K_v} \nonumber
  \intertext{Solve for the angular acceleration. First, we'll rearrange the
    terms because from inspection, $V$ is the \gls{model} \gls{input},
    $\omega_m$ is the \gls{state}, and $\tau_m$ contains the angular
    acceleration.}
  V &= \frac{R}{K_t} \tau_m + \frac{1}{K_v} \omega_m \nonumber
  \intertext{Solve for $\tau_m$.}
  V &= \frac{R}{K_t} \tau_m + \frac{1}{K_v} \omega_m \nonumber \\
  \frac{R}{K_t} \tau_m &= V - \frac{1}{K_v} \omega_m \nonumber \\
  \tau_m &= \frac{K_t}{R} V - \frac{K_t}{K_v R} \omega_m \nonumber
  \intertext{Since $\tau_m G = \tau_f$ and $\omega_m = G \omega_f$,}
  \left(\frac{\tau_f}{G}\right) &= \frac{K_t}{R} V -
    \frac{K_t}{K_v R} (G \omega_f) \nonumber \\
  \frac{\tau_f}{G} &= \frac{K_t}{R} V - \frac{G K_t}{K_v R} \omega_f \nonumber
    \\
  \tau_f &= \frac{G K_t}{R} V - \frac{G^2 K_t}{K_v R} \omega_f \label{eq:tau_f}
  \intertext{The torque applied to the flywheel is defined as}
  \tau_f &= J \dot{\omega}_f \label{eq:tau_f_def}
  \intertext{where $J$ is the moment of inertia of the flywheel and
    $\dot{\omega}_f$ is the angular acceleration. Substitute equation
    \eqref{eq:tau_f_def} into equation \eqref{eq:tau_f}.}
  (J \dot{\omega}_f) &= \frac{G K_t}{R} V - \frac{G^2 K_t}{K_v R} \omega_f
    \nonumber \\
  \dot{\omega}_f &= \frac{G K_t}{RJ} V - \frac{G^2 K_t}{K_v RJ} \omega_f
    \nonumber \\
  \dot{\omega}_f &= -\frac{G^2 K_t}{K_v RJ} \omega_f + \frac{G K_t}{RJ} V
    \nonumber
  \intertext{We'll relabel $\omega_f$ as $\omega$ for convenience.}
  \dot{\omega} &= -\frac{G^2 K_t}{K_v RJ} \omega + \frac{G K_t}{RJ} V
    \label{eq:dot_omega_flywheel}
\end{align}

This model will be converted to state-space notation in section
\ref{sec:ss_model_flywheel}.

\subsection{Calculating constants}

\subsubsection{Moment of inertia J}

Given the simplicity of this mechanism, it may be easier to compute this value
theoretically using material properties in CAD. A procedure for measuring it
experimentally is presented below.

First, rearrange equation \eqref{eq:dot_omega_flywheel} into the form
$y = mx + b$ such that $J$ is in the numerator of $m$.
\begin{align}
  \dot{\omega} &= -\frac{G^2 K_t}{K_v RJ} \omega + \frac{G K_t}{RJ} V \nonumber
    \\
  J\dot{\omega} &= -\frac{G^2 K_t}{K_v R} \omega + \frac{G K_t}{R} V \nonumber
  \intertext{Multiply by $\frac{K_v R}{G^2 K_t}$ on both sides.}
  \frac{J K_v R}{G^2 K_t} \dot{\omega} &= -\omega + \frac{G K_t}{R} \cdot
    \frac{K_v R}{G^2 K_t} V \nonumber \\
  \frac{J K_v R}{G^2 K_t} \dot{\omega} &= -\omega + \frac{K_v}{G} V \nonumber \\
  \omega &= -\frac{J K_v R}{G^2 K_t} \dot{\omega} + \frac{K_v}{G} V
    \label{eq:flywheel_J_regression}
\end{align}

The test procedure is as follows.
\begin{enumerate}
  \item Run the flywheel forward at a constant voltage. Record the angular
    velocity over time.
  \item Compute the angular acceleration from the angular velocity data as the
    difference between each sample divided by the time between them.
  \item Perform a linear regression of angular velocity versus angular
    acceleration. The slope of this line has the form $-\frac{J K_v R}{G^2 K_t}$
    as per equation \eqref{eq:flywheel_J_regression}.
  \item Multiply the slope by $-\frac{G^2 K_t}{K_v R}$ to obtain a least squares
    estimate of $J$.
\end{enumerate}
