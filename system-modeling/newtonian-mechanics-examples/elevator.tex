\section{Elevator}

This elevator consists of a DC motor attached to a pulley that drives a mass up
or down.
\begin{bookfigure}
  \input{figs/elevator-system-diagram}
  \caption{Elevator system diagram}
  \label{fig:elevator}
\end{bookfigure}

Gear ratios are written as output over input, so $G$ is greater than one in
figure \ref{fig:elevator}.

\subsection{Equations of motion}

We want to derive an equation for the carriage acceleration $a$ (derivative of
$v$) given an input voltage $V$, which we can integrate to get carriage velocity
and position.

First, we'll find a torque to substitute into the equation for a DC motor. Based
on figure \ref{fig:elevator}
\begin{equation}
  \tau_m G = \tau_p \label{eq:elevator_tau_m_ratio}
\end{equation}

where $G$ is the gear ratio between the motor and the pulley and $\tau_p$ is the
torque produced by the pulley.
\begin{equation}
  rF_m = \tau_p \label{eq:elevator_torque_pulley}
\end{equation}

where $r$ is the radius of the pulley. Substitute equation
\eqref{eq:elevator_tau_m_ratio} into $\tau_m$ in the DC motor equation
\eqref{eq:motor_tau_V}.
\begin{align}
  V &= \frac{\frac{\tau_p}{G}}{K_t} R + \frac{\omega_m}{K_v} \nonumber \\
  V &= \frac{\tau_p}{GK_t} R + \frac{\omega_m}{K_v} \nonumber
  \intertext{Substitute in equation \eqref{eq:elevator_torque_pulley} for
    $\tau_p$.}
  V &= \frac{rF_m}{GK_t} R + \frac{\omega_m}{K_v} \label{eq:elevator_Vinter1}
\end{align}

The angular velocity of the motor armature $\omega_m$ is
\begin{equation}
  \omega_m = G \omega_p \label{eq:elevator_omega_m_ratio}
\end{equation}

where $\omega_p$ is the angular velocity of the pulley. The velocity of the mass
(the elevator carriage) is
\begin{equation*}
  v = r \omega_p
\end{equation*}
\begin{equation}
  \omega_p = \frac{v}{r} \label{eq:elevator_omega_p}
\end{equation}

Substitute equation \eqref{eq:elevator_omega_p} into equation
\eqref{eq:elevator_omega_m_ratio}.
\begin{equation}
  \omega_m = G \frac{v}{r} \label{eq:elevator_omega_m}
\end{equation}

Substitute equation \eqref{eq:elevator_omega_m} into equation
\eqref{eq:elevator_Vinter1} for $\omega_m$.
\begin{align}
  V &= \frac{rF_m}{GK_t} R + \frac{G \frac{v}{r}}{K_v} \nonumber \\
  V &= \frac{RrF_m}{GK_t} + \frac{G}{rK_v} v \nonumber
  \intertext{Solve for $F_m$.}
  \frac{RrF_m}{GK_t} &= V - \frac{G}{rK_v} v \nonumber \\
  F_m &= \left(V - \frac{G}{rK_v} v\right) \frac{GK_t}{Rr} \nonumber \\
  F_m &= \frac{GK_t}{Rr} V - \frac{G^2K_t}{Rr^2 K_v} v
\end{align}

We need to find the acceleration of the elevator carriage. Note that
\begin{equation}
  \sum F = ma
\end{equation}

where $\sum F$ is the sum of forces applied to the elevator carriage, $m$ is the
mass of the elevator carriage in kilograms, and $a$ is the acceleration of the
elevator carriage.
\begin{align}
  ma &= F_m \nonumber \\
  ma &= \left(\frac{GK_t}{Rr} V - \frac{G^2K_t}{Rr^2 K_v} v\right) \nonumber \\
  a &= \frac{GK_t}{Rrm} V - \frac{G^2K_t}{Rr^2 mK_v} v \nonumber \\
  a &= -\frac{G^2K_t}{Rr^2 mK_v} v + \frac{GK_t}{Rrm} V  \label{eq:elevator_accel}
\end{align}
\begin{remark}
  Gravity is not part of the modeled dynamics because it complicates the
  state-space \gls{model} and the controller will behave well enough without it.
\end{remark}

This model will be converted to state-space notation in section
\ref{sec:ss_model_elevator}.
